\chapter{The two dimensional case}
\label{chap:bord2}

\section{Generators and relations}

We now consider the category $\bord{2}$.
Recall that its objects are closed oriented 1-dimensional manifold and its arrows are the equivalence classes of oriented 2-dimensional bordisms.
Our goal is to describe its elementary pieces and how they interact with each other. 
To do so, we are going to define a \emph{normal form} of such bordisms, which will facilitate their comparison.
%This construction relies on some results from Morse theory, which we now briefly recall. 

We begin by adressing the existence of a bordism between closed oriented 1-dimensional manifolds.
It is a known result that all closed connected one dimensional manifolds are diffeomorphic to a circle $\mathbb{S}^1$. 
Any object in $\bord{2}$ is then diffeomorphic to a disjoint union of circles.
This doesn't depend on the orientation (either clockwise or counter-clockwise) we define on the circles.
Indeed, given two copies of $\mathbb{S}^1$ with opposite orientations, we can always define, by reflection, an orientation preserving diffeomorphism between them.

