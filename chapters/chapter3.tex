\chapter{The two dimensional case}
\label{chap:bord2}

\section{Generators and relations}
orrori vari...

We now consider the category $\bord{2}$.
Recall that its objects are closed oriented 1-dimensional manifolds  and its arrows are the equivalence classes of oriented 2-dimensional bordisms.
Our goal is to describe its elementary pieces and how they interact with each other. 
To do so, we are going to define a \emph{normal form} of such bordisms, which will facilitate their comparison.
%This construction relies on some results from Morse theory, which we now briefly recall. 

We begin by adressing the existence of a bordism between closed oriented 1-dimensional manifolds.
As for the topological underlying structure, it is a known result that all closed connected one dimensional manifolds are diffeomorphic to a circle $\mathbb{S}^1$. Any object in $\bord{2}$ is then diffeomorphic to a disjoint union of circles (each with its own orientation).
Given two 1-dimensional manifolds, $\Sigma_0$ with $m$ connected components and $\Sigma_1$ with $n$ connected components, we can define a bordism between the two by considering the 2-dimensional manifold $M$ comprising of $m+n$ "spheres with a hole".

\begin{tcbprp}[Existence of 2D Bordisms]
\label{prop:existence-2d-bordisms}
For any two closed oriented 1-dimensional manifolds, there exists an oriented bordisms between them.
\end{tcbprp}


\begin{remark}
While this result may seem straightforward in dimension 2, it is not trivial in general.
For example, when considering oriented 0-manifolds, there exists a bordisms between two of them if and only if the "sums of the signed points" are equal between the two.
\todo{scribbi meglio sta robaccia}
\end{remark}

As defined, the category $\bord{2}$ contains many diffeomorphic but distinct objects, which become unnecessary \todo{maybe better "redundant"??} when studying its structure. To get a more essential representation, we take its skeleton.

\begin{tcbdfn}[Skeletons of a category]
Let $\cat{C}$ be a category. A \emph{skeleton}\index{skeleton} of $\cat{C}$ is any full subcategory $\cat{S}$ such that each object of $\cat{C}$ is isomorphic in $\cat{C}$ to \emph{exactly} one object of $\cat{S}$.
\end{tcbdfn} 
 %(equivalentely, $\cat{A}$ is the full subcategory comprised of exactly one object from each isomorphism class)
\begin{dfnx*}[Properties of skeletons]
Any two skeletons $\cat{S}$, $\cat{S}'$ of the same category $\cat{C}$ are always isomorphic. We'll then speak of \emph{the} skeleton of a category.
The inclusion $\cat{S} \hookrightarrow \cat{C}$ defines an equivalence of categories. 
We have that two categories are equivalent if and only if their skeletons are isomorphic.
\end{dfnx*}

In the case of $\bord{2}$ we firstly need to understand how the isomorphism classes of its objects are defined.

\begin{dfnx*}[Inverse bordism]
Let $M \colon \Sigma_0 \to \Sigma_1$ be an $n$-dimensional oriented bordism. The bordism $M^{-1} \colon \Sigma_1 \to \Sigma_0$ is an inverse to $M$ if $MM^{-1}$ is the identity bordism on $\Sigma_0$ and $M^{-1}M$ is the identity bordism on $\Sigma_1$.
\end{dfnx*}

%\begin{tcblemma}
%    Let $M \colon \Sigma_0 \to \Sigma_1$ be an invertible bordism. Then $\Sigma_0$ and $\Sigma_1$ have the same number of connected components.
%\end{tcblemma}

\begin{tcblemma}
    Let $\Sigma_0$, $\Sigma_1$ be two closed oriented 1-dimensional manifolds. We have that $\Sigma_0$ and $\Sigma_1$ are diffeomorphic if and only if there exists an invertible morphism between them.
\end{tcblemma}
\todo[inline]{a me sembra una cosa un po' buttata là. andrà dimostrata...?}

Again, we recall that every object of $\bord{2}$ is diffeomorphic to a disjoint union of circles, where each components carries its own orientation (either clockwise or counter-clockwise). 
However, given two copies of $\mathbb{S}^1$ with opposite orientations we can always define, by reflection, an orientation preserving diffeomorphism between them. 
By the previous lemma this implies they belong to the same iso class.
Therefore when defining the skeleton of $\bord{2}$, the orientation becomes irrelevant. The only invariant for an object is its number of connected components.

\begin{tcbdfn}[Skeleton of $\bord{2}$]
Let $(\mathbf{n})$ denote the disjoint union of $n$ copies of $\mathbb{S}^1$. Then the objects 
\[
\{(\mathbf{0}), (\mathbf{1}), (\mathbf{2}), (\mathbf{3}), \dots\} = \{\varnothing, \mathbb{S}^1, \mathbb{S}^1 \disj \mathbb{S}^1, \mathbb{S}^1 \disj \mathbb{S}^1 \disj \mathbb{S}^1, \dots\} %= \{\varnothing, \Circle, \Circle \disj \Circle, \Circle \disj \Circle \disj \Circle, \dots\}
\]
together with all the possible morphisms between them, form a skeleton of $\bord{2}$. By abuse of notation, from now on, we we'll refer to this category as $\bord{2}$.
\end{tcbdfn}

% mmmm
We now analyze the morphisms in detail.
\begin{tcbprp}
Two diffeomorphisms from $\Sigma_0$ to $\Sigma_1$ define the same bordism equivalence class if and only if they are (smoothly) homotopic.
\end{tcbprp}

In dimension 1, the classification of diffeomorphisms up to homotopy is particularly simple. It can be proved that every orientation preserving diffeomorphism from $\mathbb{S}^1$ to $\mathbb{S}^1$ is homotopic to the identity. The only invertible 2-bordisms are then permutation bordisms
\todo[inline]{non certa di mettere questa proposizione. rivedere e in caso definire i permutation bordisms}
% mmmm

\begin{tcbdfn}[Generators for a monoidal category]
    A \emph{generating set} for a monoidal category $(\cat{C}, \square, I)$ is a set of arrows $S \subset \arr{\cat{C}}$ such that every arrow in $\cat{C}$ can be obtained from the ones in $S$ by composition or through the monoidal product $\square$.
\end{tcbdfn}

In the case of 2-dimensional bordisms, for example, the identity over the disjoint union of two circles ($\id_{(\mathbf{2})}$) is not a generator, since it decomposes as the disjoint union of two identities over a circle ($\id_{(\mathbf{1})}\disj\id_{(\mathbf{1})}$). On the contrary, the twist bordism $T$ cannot be described as a disjoint union of identity boridsms. The exchange of components described by the twist makes it impossible to define a diffeomorphism to the identity on $\mathbb{S}^1 \disj \mathbb{S}^1$.

\begin{tcbthm}[Generators of $\bord{2}$]
The six bordisms 
\[
\begin{tikzpicture}[every tqft/.style={transform shape}]
\pic[tqft cap, rotate=90, draw, at={(0,0)},
every outgoing lower boundary component/.style={draw},  every incoming lower boundary component/.style={dashed, draw}
];
\pic[tqft pair of pants, rotate=90, draw, at={(3,0)},
every outgoing lower boundary component/.style={draw},  every incoming lower boundary component/.style={dashed, draw}
];
\pic[tqft cup, rotate=90, draw, at={(6,0)},
every outgoing lower boundary component/.style={draw},  every incoming lower boundary component/.style={dashed, draw}
];
\pic[tqft reverse pair of pants, rotate=90, draw, at={(7,-1)},
every outgoing lower boundary component/.style={draw},  every incoming lower boundary component/.style={dashed, draw}
];
\pic[tqft cylinder to prior, rotate=90, draw, at={(10,.5)},
every outgoing lower boundary component/.style={draw},  every incoming lower boundary component/.style={dashed, draw}
];
\pic[tqft cylinder to next, rotate=90, draw, at={(10,-0.5)},
every outgoing lower boundary component/.style={draw},  every incoming lower boundary component/.style={dashed, draw}
];
\pic[tqft cylinder, rotate=90, draw, at={(13,0)},
every outgoing lower boundary component/.style={draw},  every incoming lower boundary component/.style={dashed, draw}
];
\end{tikzpicture}
\]
form a generating set for the monoidal category $(\bord{2}, \disj, \varnothing)$.
\end{tcbthm}

The proof of this theorem requires some Morse theory. We recall the essential tools.
\begin{tcbdfn}[Critical points]
    Let $M$ be a compact $n$-dimensional manifold and consider a smooth map $f \colon M \to I \subset \mathbb{R}$. A point $x \in M$ is a \emph{critical point} if the differential $df_x$ is zero. In such case, its image $f(x) \in I$ becomes a critical value.
\end{tcbdfn}
\begin{tcbdfn}[Non-degeneracy of critical points]
    Let $M$ be a compact $n$-dimensional manifold. A critical point $x \in M$ is \emph{non degenerate} if the Hessian matrix has non zero determinant in some coordinates. The \emph{index} 
\end{tcbdfn}
\begin{tcbdfn}[Morse functions]
    bla
\end{tcbdfn}
\begin{tcbthm}
    bla bla
\end{tcbthm}
We now give a non rigorous sketch of the main argument.



Suppose M is a smooth manifold that admits a proper smooth function
f W M ! R with no critical points. Show that M is diffeomorphic to N R
for some compact smooth manifold N .