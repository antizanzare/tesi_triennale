\chapter{The two dimensional case}
\label{chap:bord2}

\section{Generators and relations}

We now consider the category $\bord{2}$.
Recall that its objects are closed oriented 1-dimensional manifolds  and its arrows are the equivalence classes of oriented 2-dimensional bordisms.
Our goal is to describe its elementary pieces and how they interact with each other. 
To do so, we are going to define a \emph{normal form} of such bordisms, which will facilitate their comparison.
%This construction relies on some results from Morse theory, which we now briefly recall. 

We begin by adressing the existence of a bordism between closed oriented 1-dimensional manifolds.
As for the topological underlying structure, it is a known result that all closed connected one dimensional manifolds are diffeomorphic to a circle $\mathbb{S}^1$. Any object in $\bord{2}$ is then diffeomorphic to a disjoint union of circles (each with its own orientation).
Given two 1-dimensional manifolds, $\Sigma_0$ with $m$ connected components and $\Sigma_1$ with $n$ connected components, we can define a bordism between the two by considering the 2-dimensional manifold $M$ comprising of $m+n$ ``spheres with a hole''.

\begin{tcbprp}[Existence of 2D Bordisms]
\label{prop:existence-2d-bordisms}
For any two closed oriented 1-dimensional manifolds, there exists an oriented bordisms between them.
\end{tcbprp}


\begin{remark}
While this result may seem straightforward in dimension 2, it is not trivial in general.
For example, when considering oriented 0-manifolds, there exists a bordisms between two of them if and only if the ``sums of the signed points'' are equal between the two.
\todo{scribbi meglio sta robaccia}
\end{remark}

As defined, the category $\bord{2}$ contains many diffeomorphic but distinct objects, which become unnecessary \todo{maybe better "redundant"??} when studying its structure. To get a more essential representation, we take its skeleton.

\begin{tcbdfn}[Skeletons of a category]
Let $\cat{C}$ be a category. A \emph{skeleton}\index{skeleton} of $\cat{C}$ is any full subcategory $\cat{S}$ such that each object of $\cat{C}$ is isomorphic in $\cat{C}$ to \emph{exactly} one object of $\cat{S}$.
\end{tcbdfn} 
 %(equivalentely, $\cat{A}$ is the full subcategory comprised of exactly one object from each isomorphism class)
\begin{dfnx*}[Properties of skeletons]
Any two skeletons $\cat{S}$, $\cat{S}'$ of the same category $\cat{C}$ are always isomorphic. We'll then speak of \emph{the} skeleton of a category.
The inclusion $\cat{S} \hookrightarrow \cat{C}$ defines an equivalence of categories. 
We have that two categories are equivalent if and only if their skeletons are isomorphic.
\end{dfnx*}

In the case of $\bord{2}$ we firstly need to understand how the isomorphism classes of its objects are defined.

\begin{dfnx*}[Inverse bordism]
Let $M \colon \Sigma_0 \to \Sigma_1$ be an $n$-dimensional oriented bordism. The bordism $M^{-1} \colon \Sigma_1 \to \Sigma_0$ is an inverse to $M$ if $MM^{-1}$ is the identity bordism on $\Sigma_0$ and $M^{-1}M$ is the identity bordism on $\Sigma_1$.
\end{dfnx*}

%\begin{tcblemma}
%    Let $M \colon \Sigma_0 \to \Sigma_1$ be an invertible bordism. Then $\Sigma_0$ and $\Sigma_1$ have the same number of connected components.
%\end{tcblemma}

\begin{tcblemma}
    Let $\Sigma_0$, $\Sigma_1$ be two closed oriented 1-dimensional manifolds. We have that $\Sigma_0$ and $\Sigma_1$ are diffeomorphic if and only if there exists an invertible morphism between them.
\end{tcblemma}
\todo[inline]{a me sembra una cosa un po' buttata là. andrà dimostrata...?}

Again, we recall that every object of $\bord{2}$ is diffeomorphic to a disjoint union of circles, where each components carries its own orientation (either clockwise or counter-clockwise). 
However, given two copies of $\mathbb{S}^1$ with opposite orientations we can always define, by reflection, an orientation preserving diffeomorphism between them. 
By the previous lemma this implies they belong to the same iso class.
Therefore when defining the skeleton of $\bord{2}$, the orientation becomes irrelevant. The only invariant for an object is its number of connected components.

\begin{tcbdfn}[Skeleton of $\bord{2}$]
Let $(\mathbf{n})$ denote the disjoint union of $n$ copies of $\mathbb{S}^1$. Then the objects 
\[
\{(\mathbf{0}), (\mathbf{1}), (\mathbf{2}), (\mathbf{3}), \dots\} = \{\varnothing, \mathbb{S}^1, \mathbb{S}^1 \disj \mathbb{S}^1, \mathbb{S}^1 \disj \mathbb{S}^1 \disj \mathbb{S}^1, \dots\} %= \{\varnothing, \Circle, \Circle \disj \Circle, \Circle \disj \Circle \disj \Circle, \dots\}
\]
together with all the possible morphisms between them, form a skeleton of $\bord{2}$. By abuse of notation, from now on, we we'll refer to this category as $\bord{2}$.
\end{tcbdfn}

We now analyze the morphisms in detail.


\begin{tcbdfn}[Generators for a monoidal category]
    A \emph{generating set} for a monoidal category $(\cat{C}, \square, I)$ is a set of arrows $S \subset \arr{\cat{C}}$ such that every arrow in $\cat{C}$ can be obtained from the ones in $S$ by composition or through the monoidal product $\square$.
\end{tcbdfn}

In the case of 2-dimensional bordisms, for example, the identity over the disjoint union of two circles ($\id_{(\mathbf{2})}$) is not a generator, since it decomposes as the disjoint union of two identities over a circle ($\id_{(\mathbf{1})}\disj\id_{(\mathbf{1})}$). On the contrary, the twist bordism $T$ cannot be described as a disjoint union of identity boridsms. The exchange of components described by the twist makes it impossible to define a diffeomorphism to the identity on $\mathbb{S}^1 \disj \mathbb{S}^1$.

\begin{tcbthm}[Generators of $\bord{2}$]
    \label{thm-generators}
The six bordisms \todo[inline]{cap, pants, cocap, copants, twist, cylinder}
\[
\begin{tikzpicture}[every tqft/.style={transform shape}]
\pic[tqft cap, rotate=90, draw, at={(0,0)},
every outgoing lower boundary component/.style={draw},  every incoming lower boundary component/.style={dashed, draw}
];
\pic[tqft pair of pants, rotate=90, draw, at={(3,0)},
every outgoing lower boundary component/.style={draw},  every incoming lower boundary component/.style={dashed, draw}
];
\pic[tqft cup, rotate=90, draw, at={(6,0)},
every outgoing lower boundary component/.style={draw},  every incoming lower boundary component/.style={dashed, draw}
];
\pic[tqft reverse pair of pants, rotate=90, draw, at={(7,-1)},
every outgoing lower boundary component/.style={draw},  every incoming lower boundary component/.style={dashed, draw}
];
\pic[tqft cylinder to prior, rotate=90, draw, at={(10,.5)},
every outgoing lower boundary component/.style={draw},  every incoming lower boundary component/.style={dashed, draw}
];
\pic[tqft cylinder to next, rotate=90, draw, at={(10,-0.5)},
every outgoing lower boundary component/.style={draw},  every incoming lower boundary component/.style={dashed, draw}
];
\pic[tqft cylinder, rotate=90, draw, at={(13,0)},
every outgoing lower boundary component/.style={draw},  every incoming lower boundary component/.style={dashed, draw}
];
\end{tikzpicture}
\]
form a generating set for the monoidal category $(\bord{2}, \disj, \varnothing)$.
\end{tcbthm}

%The proof of this theorem relies on Morse theory. 
%To avoid losing the main focus, we only state the main tecnichal result needed and we refer to [Hirsch] \todo{add citation} for the foundational definitions and a complete treatment of the underlying theory.


%\begin{tcbthm}
%    \label{thm-saddle}
%    Let $M$ be a compact connected orientable surface with Morse function $f \colon M \to I$. 
%    If $f$ has a unique critical point $x$, of index 1, then $M$ is diffeomorphic to the pair of pants. 
%    If, instead, $f$ has no critical points, $M$ is equivalent to a cylinder.
%\end{tcbthm}
%We now give a non rigorous sketch of the main argument. By definition of symmetric monoidal category, we clearly have the identity and the twist as generating bordisms. Now, consider any bordism $M\ \colon \Sigma_0 \to \Sigma_1$. We take a Morse function $f\colon M \to [0,1]$ and a sequence of regular values $0 = r_0 \leq r_1 \leq r_2 \leq \dots \leq r_k = 1$ such that each interval $[r_i, r_{i+1}]$ contains \emph{at most} one critical value. By considering the preimages of the regular values $\Sigma_i \defeq f^{-1}(i)$ we decompose $M$ in elementary bordisms $M_{[i,i+1]} \colon \Sigma_i \to \Sigma_{i+1}$, each of which with at most one critical point.
%We can assume (through twist and disjoint union) each of these bordisms is connected and has unique critical point $x$. The local structure of each $M_{[i,i+1]}$ is determined by the index of such critical point:
%\begin{itemize}
%    \item if the index is 0, $x$ is a local minimum and the bordism is a cap;
%    \item if the index is 2, $x$ is a local maximum and the bordism is a cocap;
%    \item if the index is 1, $x$ is a saddle and by theorem \ref{thm-saddle} the borism is a pair of pants (or a copair).
%\end{itemize}

%Having established the generators of $\bord{2}$, we now want to describe their interactions by providing a sufficient set of relations. To do so we are going to define a \emph{normal form} and define a set of relations sufficient to reduce any bordism into this standard form.

The proof of this theorem relies on Morse theory, for which we refer the reader to [Hirsch, ch 6 and 9]\todo{add citation}. 
However, in dimension 2, we can also appeal to the topological classification of surfaces, which provides a more elementary construction. 
To avoid losing the main focus, we will follow this second approach. 
In higher dimensions this is not possilble and we will need to rely on Morse theory.

We recall the classical classification of surfaces without boundary.
\par
\emph{Two connected compact oriented surfaces without boundary are diffeomorphic if and only if they have the same genus (or equivalentely the same Euler characteristic).}
\par
Since bordisms are surfaces \emph{with} boundary we require an adapted version of the result. 
\begin{dfnx*}[Topological calssification of surfaces with boundary]
Two connected compact oriented surfaces with oriented boundary are diffeomorphic if and only if they have the same genus (or equivalentely the same Euler characteristic), the same number of in-boundaries and the same number of out-boundaries.
\end{dfnx*}

To keep track of such invariants we define the normal form in the following way.

\begin{tcbdfn}[Normal form of connected bordisms]
We define the normal form of a connected bordism $M \colon (\mathbf{m}) \to (\mathbf{n})$ with $g$ holes as the composition of
\begin{itemize}
    \item an \emph{in-part} $M_{in} \colon (\mathbf{m}) \to (\mathbf{1})$, consisting of $m-1$ copies of copair of pants $\bordcopants$. If $m = 0$, the in-part consists of a single cap $\bordcap$.
    \item a \emph{topological part} $M_{mid} \colon (\mathbf{1}) \to (\mathbf{1})$, consisting of $g$ pair of pants and $g$ copair of pants, arranged together to define $g$ holes $\bordpants \bordcopants$.
    \item an \emph{out-part} $M_{out} \colon (\mathbf{1}) \to (\mathbf{n})$, consisting of $n-1$ copies of pair of pants $\bordpants$. If $n = 0$, the in-part consists of a single cocap $\bordcocap$.
\end{itemize}
\end{tcbdfn}
%($\bordcap$)($\bordcocap$)($\bordpants$)($\bordcopants$)($\bordcylinder$)($\bordtwist$)
To understand how we are supposed to attach such bordisms let's see a visual example.
\todo[inline]{addare la 'vera' decomposizione}
\begin{figure}[h]
    \centering
    \begin{tikzpicture}[every tqft/.style={transform shape}, rotate=90]
    \pic[name=a, 
        tqft, 
        incoming boundary components = 5, 
        outgoing boundary components = 4,
        genus = 4,
        draw,
        every outgoing lower boundary component/.style={draw},
        every incoming lower boundary component/.style={dashed, draw},
        boundary separation=1cm,
        cobordism height=3cm,
        offset=0.5
    ];
    \end{tikzpicture}
\end{figure}

\begin{tcblemma}
    Every connected 2-dimensional bordism can be obtained through composition and disjoint union of the elementary bordisms $\bordcap$, $\bordcocap$, $\bordpants$, $\bordcopants$, $\bordcylinder$.
\end{tcblemma}

Let us now consider a non connected bordisms. While it is true that a non connected manifold is the disjoint union of connected surfaces, this fact doesn't help us in proving \ref{thm-generators}. Indeed the category of bordisms not only considers the topology of bordisms but also the permutations on the boundaries.
The main example of this fact can be seen by considering the twist. As a smooth manifold, the twist is the disjoint union of two cylinders. However, as a bordism, it is not isomorphic to the disjoint union of two identity morphisms since it flips the two boundaries.
This problem can be solved by introducing permutations that let us untwist the bordisms. We now describe such procedure.

Let $M \colon (\mathbf{m}) \to (\mathbf{n})$ be a non connected bordism from $\Sigma_1 \disj \dots \disj \Sigma_m$ to $\Sigma'_1 \disj \dots \disj \Sigma'_n$. For simplicity we assume $M$ only has two connected components which we name $M_1$ and $M_2$. Some of the inboundaries of $M$ will then be inboundaries of $M_1$, while the others will be in boundaries of $M_2$. 
We ``reorder'' such boundaries through a diffeomorphism $(\mathbf{m}) \to (\mathbf{m})$, which induces a bordism $S \colon (\mathbf{m}) \to (\mathbf{m})$. 
We can now take the composition $SM$ and denote as $(SM)_1$ and $(SM)_2$ its two connected components. By construction, the in-boundary of $SM$ is exactly the disjoint union of the in-boundaries of $(SM)_1$, $(SM)_2$.
By applying the same method to the out-boundary we get a permutation bordism $T \colon (\mathbf{n}) \to (\mathbf{n})$. 
The composition $SMT \colon (\mathbf{m}) \to (\mathbf{n})$ is then exactly the disjoint union of its connected components.
The intial bordism $M$ then factors as $S^{-1} (SMT) T^{-1}$, letting us state the following.

\begin{tcblemma}
Every 2-dimensional bordism factors as a permutation bordism, followed by a disjoint union of connected bordisms, followed by a permutation bordism.
\end{tcblemma}



% mmmm
%\todo[inline]{non certa di mettere questa proposizione. rivedere e in caso definire i permutation bordisms}
%\begin{tcbprp}
%Two diffeomorphisms from $\Sigma_0$ to $\Sigma_1$ define the same bordism equivalence class if and only if they are (smoothly) homotopic.
%\end{tcbprp}
%In dimension 1, the classification of diffeomorphisms up to homotopy is particularly simple. It can be proved that every orientation preserving diffeomorphism from $\mathbb{S}^1$ to $\mathbb{S}^1$ is homotopic to the identity. The only invertible 2-bordisms are then permutation bordisms
% mmmm

