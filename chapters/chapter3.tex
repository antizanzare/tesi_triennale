\chapter{The two dimensional case}
\label{chap:bord2}

\section{Generators and relations}
orrori vari...

We now consider the category $\bord{2}$.
Recall that its objects are closed oriented 1-dimensional manifolds  and its arrows are the equivalence classes of oriented 2-dimensional bordisms.
Our goal is to describe its elementary pieces and how they interact with each other. 
To do so, we are going to define a \emph{normal form} of such bordisms, which will facilitate their comparison.
%This construction relies on some results from Morse theory, which we now briefly recall. 

We begin by adressing the existence of a bordism between closed oriented 1-dimensional manifolds.
As for the topological underlying structure, it is a known result that all closed connected one dimensional manifolds are diffeomorphic to a circle $\mathbb{S}^1$. Any object in $\bord{2}$ is then diffeomorphic to a disjoint union of circles (each with its own orientation).
Given two 1-dimensional manifolds, $\Sigma_0$ with $m$ connected components and $\Sigma_1$ with $n$ connected components, we can define a bordism between the two by considering the 2-dimensional manifold $M$ comprising of $m+n$ "spheres with a hole".

\begin{tcbprp}[Existence of 2D Bordisms]
\label{prop:existence-2d-bordisms}
For any two closed oriented 1-dimensional manifolds, there exists an oriented bordisms between them.
\end{tcbprp}


\begin{remark}
While this result may seem straightforward in dimension 2, it is not trivial in general.
For example, when considering oriented 0-manifolds, there exists a bordisms between two of them if and only if the "sums of the signed points" are equal between the two.
\todo{scribbi meglio sta robaccia}
\end{remark}

As defined, the category $\bord{2}$ contains many diffeomorphic but distinct objects, which become unnecessary \todo{maybe better "redundant"??} when studying its structure. To get a more essential representation, we take its skeleton.

\begin{tcbdfn}[Skeletons of a category]
Let $\cat{C}$ be a category. A \emph{skeleton}\index{skeleton} of $\cat{C}$ is any full subcategory $\cat{S}$ such that each object of $\cat{C}$ is isomorphic in $\cat{C}$ to \emph{exactly} one object of $\cat{S}$.
\end{tcbdfn} 
 %(equivalentely, $\cat{A}$ is the full subcategory comprised of exactly one object from each isomorphism class)
\begin{dfnx*}[Properties of skeletons]
Any two skeletons $\cat{S}$, $\cat{S}'$ of the same category $\cat{C}$ are always isomorphic. We'll then speak of \emph{the} skeleton of a category.
The inclusion $\cat{S} \hookrightarrow \cat{C}$ defines an equivalence of categories. 
We have that two categories are equivalent if and only if their skeletons are isomorphic.
\end{dfnx*}

In the case of $\bord{2}$ we firstly need to understand how the isomorphism classes of its objects are defined.

\begin{dfnx*}[Inverse bordism]
Let $M \colon \Sigma_0 \to \Sigma_1$ be an $n$-dimensional oriented bordism. The bordism $M^{-1} \colon \Sigma_1 \to \Sigma_0$ is an inverse to $M$ if $MM^{-1}$ is the identity bordism on $\Sigma_0$ and $M^{-1}M$ is the identity bordism on $\Sigma_1$.
\end{dfnx*}

%\begin{tcblemma}
%    Let $M \colon \Sigma_0 \to \Sigma_1$ be an invertible bordism. Then $\Sigma_0$ and $\Sigma_1$ have the same number of connected components.
%\end{tcblemma}

\begin{tcblemma}
    Let $\Sigma_0$, $\Sigma_1$ be two closed oriented 1-dimensional manifolds. We have that $\Sigma_0$ and $\Sigma_1$ are diffeomorphic if and only if there exists an invertible morphism between them.
\end{tcblemma}
\todo[inline]{a me sembra una cosa un po' buttata là. andrà dimostrata...?}

We hence recall that every object of $\bord{2}$ is diffeomorphic to a disjoint union of circles. We  