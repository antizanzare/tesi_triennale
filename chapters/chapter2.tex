\chapter{Bordisms and TQFTs}
\label{chap:bord}

We denote with $\mathbb{R}^n$ the $n$-dimensional Euclidian space composed of all ordered $n$-tuples $(x_1,.....,x_n)$ of real numbers. 
We denote with $\mathbb{H}^n$ the set $\{ x \in \mathbb{R}^n \colon x_n \geq 0 \}$ equipped with the topology induced by $\mathbb{R}^n$. 
Recall that an $n$-dimensional \emph{topological manifold} is a second-countable Hausdorff topological space locally homeomorphic to $\mathbb{R}^n$.
An $n$-dimensional \emph{topological manifold with boundary} is a second-countable Hausdorff topological space locally homeomorphic to $\mathbb{H}^n$.
When saying we have a \emph{local chart} on a topological manifold $M$, we are referring to a pair $(U, u)$ where $U \subseteq X$ is an open subset of $X$ and $u \colon U \to \underline{U} \subseteq \mathbb{R}^n$ is an homeomorphism.
Two intersecting charts $(U,u)$, $(V,v)$ on $M$ are \emph{($C^\infty$) compatible} if the transition maps $u \circ v^{-1}$ and $v \circ u^{-1}$ are infinitely differentiable.
An \emph{atlas} $\mathcal{A}$ on $M$ is a collection $\{(U_i, u_i)\}_{i \in I}$ of compatible charts such that $\{U_i\}_{i \in I}$ is an open covering of $M$.
Two atlases $\mathcal{A}$, $\mathcal{B}$ are \emph{compatible} if $\mathcal{A} \cup \mathcal{B}$ is an atlas.
A \emph{smooth ($C^\infty$) manifold} is a second-countable Hausdorff topological space $M$ equipped with a \emph{smooth structure}, i.e. an equivalence class of compatible atlases.
In the same vein we can define a \emph{smooth manifold with boundary}. 
A point $x \in M$ is a \emph{boundary point} if it is mapped, through some local chart, into the boundary $\{ x \in \mathbb{R}^n \colon x_n = 0\}$ of $\mathbb{H}^n$. 
The set of all boundary points of an $n$-dimensional mandifold $M$ is an $(n-1)$-dimensional manifold we'll denote with $\partial M$, the \emph{boundary} of $M$.
The boundary of a manifold can also be empty. In such way every manifold can be seen as a manifold with boundary.
Finally, we define a \emph{closed manifold} as a compact manifold with no boundary. 

\section{Bordisms}

Having recalled the basic framework, we continue with some more precise notions from differential geometry. 
We'll assume all manifolds to be smooth ones.

\begin{tcbdfn}[Orientation of a vector space]
Let $V$ be a finite dimensional real vector space. We say that two ordered basis $\mathcal{B}_1$, $\mathcal{B}_2$ have the \emph{same orientation} (resp. \emph{opposite orientation} if the linear transformation carrying one to the other has positive (resp. negative) determinant.
An \emph{orientation} on $V$ is given by associating a sign (either $+$ or $-$) to each ordered basis, following the rule just stated.  
\end{tcbdfn}

\begin{tcbdfn}[Linear maps and orientation]
Let $V$, $W$ be two ordered finite dimensional real vector spaces. A linear map $f \colon V \to W$ is \emph{orientation preserving} if it sends positive basis into positive basis and \emph{orientation reversing} if positive bases are sent into negative ones.
\end{tcbdfn}

\begin{tcbdfn}[Oriented manifolds]
An \emph{orientation} of a (smooth) manifold $M$ is a choice of orientation of its tangent bundle $TM$. 
In making such choice we require that the differentials of the transition functions preserve orientations. 
We say a smooth manifold $M$ is \emph{orientable} when it admits an orientation.
\end{tcbdfn}
\todo[inline]{add the notion of tangent bundle in the "recall"}

Each orientable connected smooth manifold admits two possible orientations. 
An orientable manifold with $k \in [0, \infty)$ connected components admits $2^k$ possible orientations.
The empty manifold has exactly one orientation.

\begin{tcbdfn}[Orientation of a product]
Let $M$, $N$ be two oriented manifolds, where at least one of them has no boundary. 
The product $M \times N$ acquires an orientation such that for each point $(x, y) \in M \times N$, if $\{v_1, \dots, v_m \}$ is a positive basis for $T_x(M)$ and $\{w_1, \dots, w_n\}$ is a positive basis for $T_y(N)$, then $\{v_1, \dots, v_m,w_1, \dots, w_n\}$ is a positive basis for $T_{(x,y)}(M \times N)$.
\end{tcbdfn}


\begin{example}
Take a circle $\mathbb{S}^1$ with the usual counterclockwise orientation and the interval $I \defeq [0,1]$ with its standard orientation. 
The products $\mathbb{S} \times I$ and $I \times \mathbb{S}$ then have opposite orientations.  
We hence need to be careful when choosing the order of such factors.
\[
\begin{tikzpicture}[every tqft/.style={transform shape}, rotate=90]
\pic[tqft,
  incoming boundary components=1,
  outgoing boundary components=1,
  draw,
  every outgoing lower boundary component/.style={draw},
  every incoming lower boundary component/.style={dashed,draw},
  name=a,
  anchor={(0,0)}
];
\pic[tqft,
  incoming boundary components=1,
  outgoing boundary components=1,
  draw,
  every outgoing lower boundary component/.style={draw},
  every incoming lower boundary component/.style={dashed,draw},
  name=b,
  anchor={(0,-3)}
];
\node at (a-between first incoming and first outgoing) [above=0.25em] {$\curvearrowleft$};
\node at (b-between first incoming and first outgoing) [above=0.25em] {$\curvearrowright$};
%\node at (2,-1) {$\xi$};
\end{tikzpicture}
\]
\end{example}

\begin{tcbdfn}
Let $M$ be a manifold with boundary and $p\in \partial M$. 
A vector $v \in T_p M \setminus T_p (\partial M)$ is \emph{inward-pointing} if for some $\varepsilon > 0$ there exists a smooth curve $\gamma \colon [0, \varepsilon) \to M$ such that $\gamma(0) = p$ and $\gamma'(0) = v$.
A vector $v \in T_p M \setminus T_p (\partial M)$ is \emph{outward-pointing} if there exists such a curve whose domain is $(-\varepsilon, 0]$.
\end{tcbdfn}

By considering some smooth chart $(U \ni p, \varphi = (x^i))$, we then have that the inward-pointing vectors in $T_p M$ are exactly the ones with $x^n > 0$ and the outward-pointing ones are those for which $x^n < 0$.

%ref: dfn 15.24 Lee
\begin{tcbdfn}[The induced orientation on a boundary]
    Let $M$ be an oriented manifold with boundary. Its boundary $\partial M$ inherits a canonical orientation, defined as follows.  
    Consider an outward pointing vector $n \in T_x M \setminus T_x (\partial M)$; for any basis $\{ t_1, \dots, t_n\}$ of $T_x (\partial M)$ we say it to be a $\emph{positive}$ (resp. \emph{negative}) if $\{n, t_1, \dots, t_n\}$ is a positive (resp. negative) basis for $T_x M$.
\end{tcbdfn}

Following such convention, the oriented cylinders above will induce opposite orientations on the two components of the boundary.

\[
\begin{tikzpicture}[every tqft/.style={transform shape}, rotate=90]
\pic[tqft,
  incoming boundary components=1,
  outgoing boundary components=1,
  draw,
  every outgoing lower boundary component/.style={draw},
  every incoming lower boundary component/.style={dashed,draw},
  every outgoing upper boundary component/.style={decorate,
    decoration={markings, mark=at position .5 with {\arrow{>}},},
  },
  every incoming upper boundary component/.style={decorate,
    decoration={markings, mark=at position .5 with {\arrowreversed{>}},},
  },
  name=a,
  anchor={(0,0)}
];
\pic[tqft,
  incoming boundary components=1,
  outgoing boundary components=1,
  draw,
  every outgoing lower boundary component/.style={draw},
  every incoming lower boundary component/.style={dashed,draw},
  every outgoing upper boundary component/.style={decorate,
    decoration={markings, mark=at position .5 with {\arrowreversed{>}},},
  },
  every incoming upper boundary component/.style={decorate,
    decoration={markings, mark=at position .5 with {\arrow{>}},},
  },
  name=b,
  anchor={(0,-3)}
];
\node at (a-between first incoming and first outgoing) [above=0.25em] {$\curvearrowleft$};
\node at (b-between first incoming and first outgoing) [above=0.25em] {$\curvearrowright$};
%\node at (2,-1) {$\xi$};
\end{tikzpicture}
\]
We are finally ready to give the following definition.
\begin{tcbdfn}[Oriented bordisms]
Let $\Sigma_0$, $\Sigma_1$ be two oriented closed manifolds of dimension $n-1$.
An \emph{$n$-dimensional oriented bordism}\index{oriented bordism} from $\Sigma_0$ to $\Sigma_1$ is a triple $(M, i_0, i_1)$ consisting of
\begin{itemize}
\item an $n$-dimensional oriented manifold $M$ with boundary,
\item two orientation preserving embeddings
\[
i_0 \colon \Sigma_0 \times [0,\varepsilon) \to M
\qquad 
i_1 \colon \Sigma_1 \times (1-\varepsilon, 1] \to M   
\] 
definining an \emph{in-boundary} $(\partial M)_0 \defeq i_0(\Sigma_0, 0)$ and an \emph{out-boundary} $(\partial M)_0 \defeq i_1(\Sigma_1, 1)$, such that $\partial M = (\partial M)_0 \sqcup (\partial M)_1$.
\end{itemize}
\end{tcbdfn}

\begin{example}
% Example dim.2
\begin{figure}[h]
\centering
\begin{tikzpicture}[every tqft/.style={transform shape}]
\pic[name=a,
    tqft,
    offset=-0.7,
    incoming boundary components=1,
    outgoing boundary components=2,
%    genus=1,
    draw,
    at={(1,0)},
    anchor=incoming boundary 1,
    every outgoing lower boundary component/.style={draw},
%    every incoming lower boundary component/.style={dashed,draw},
    incoming upper boundary component 1/.style={decorate, decoration={markings, mark=at position .5 with {$\arrowreversed{>}$},},},
    every outgoing upper boundary component/.style={decorate, decoration={markings, mark=at position .5 with {$\arrow{>}$},},},
    ];
\draw[dashed] ($(a-incoming boundary 1) + (0,-0.2)$) circle [x radius=10pt, y radius=5pt];
\draw[dashed] ($(a-outgoing boundary 1) + (0.02,0.2)$) circle [x radius=10pt, y radius=5pt];
\draw[dashed] ($(a-outgoing boundary 2) + (-0.02,0.2)$) circle [x radius=10pt, y radius=5pt];
\pic[tqft,
  rotate=90,
  incoming boundary components=1,
  outgoing boundary components=1,
  cobordism height=0.2cm,
  every outgoing lower boundary component/.style={draw},
  every incoming lower boundary component/.style={dashed,draw},
  every outgoing upper boundary component/.style={draw,
    decoration={markings, mark=at position .5 with {\arrow{>}},},
    postaction=decorate
  }, 
  every incoming upper boundary component/.style={dashed, draw,
    decoration={markings, mark=at position .5 with {\arrowreversed{>}}},
    postaction=decorate},
  cobordism edge/.style={draw},
%  every incoming upper boundary component/.style={dashed,draw},
  name=a,
  anchor=incoming boundary 1,
  at={(5,-1)}
  ];

  \pic[tqft,
  rotate=90,
  incoming boundary components=1,
  outgoing boundary components=1,
  cobordism height=0.2cm,
  every outgoing lower boundary component/.style={dashed,draw},
  every incoming lower boundary component/.style={draw},
  every outgoing upper boundary component/.style={dashed, draw,
    decoration={markings, mark=at position .5 with {\arrow{>}},},
    postaction=decorate
  }, 
  every incoming upper boundary component/.style={draw,
    decoration={markings, mark=at position .5 with {\arrowreversed{>}}},
    postaction=decorate},
  cobordism edge/.style={draw},
%  every incoming upper boundary component/.style={dashed,draw},
  name=a,
  anchor=incoming boundary 1,
  at={(-4,0)}
  ];
  \pic[tqft,
  rotate=90,
  incoming boundary components=1,
  outgoing boundary components=1,
  cobordism height=0.2cm,
  every outgoing lower boundary component/.style={dashed,draw},
  every incoming lower boundary component/.style={draw},
  every outgoing upper boundary component/.style={dashed, draw,
    decoration={markings, mark=at position .5 with {\arrow{>}},},
    postaction=decorate
  }, 
  every incoming upper boundary component/.style={draw,
    decoration={markings, mark=at position .5 with {\arrowreversed{>}}},
    postaction=decorate},
  cobordism edge/.style={draw},
%  every incoming upper boundary component/.style={dashed,draw},
  name=a,
  anchor=incoming boundary 1,
  at={(-4,-2)}
  ];
\end{tikzpicture}
\end{figure}
\todo[inline]{add "incoming/outcoming" and embeddings}
\end{example}

The role of the embeddings $i_0$, $i_1$ in the above definition is to identify a collar of the in-boundary and the out-boundary. 
To make this precise, we again go back to differential geometry \cite{smooth}. 
\begin{tcbdfn}[Collars]
Let $M$ be a manifold with boundary. 
A neighbourhood of $\partial M$ is called a \emph{collar neighbourhood}\index{collar neighbourhood} if it is the image of a smooth embedding $\partial M \times [0, \varepsilon) \to M$, that restircts to the identification $\partial M \times \{0\} \to \partial M$.
\end{tcbdfn}

\begin{tcbthm}[Collar neighbourhood theorem]
Let $M$ be a smooth manifold with nonempty boundary. Then $\partial M$ has a collar neighbourhood.
\end{tcbthm}

This guarantees that the notion of an oriented bordism is well defined. 

\begin{tcbthm}[Gluing smooth manifolds along their boundaries]\label{th:gluing}
Let $M$, $N$ be two manifolds with non empty boundaries $\partial M$, $\partial N$. 
Suppose we have a diffeomorphism between the two boundaries and consider the topological pushout $M \sqcup_{\partial M \iso \partial N} N$.
Such topological manifold then has a smooth structure, compatible with the smooth structures on $M$ and $N$. 
If $M$, $N$ are both compact, $M \sqcup_{\partial M \iso \partial N} N$ is compact.
If $M$, $N$ are both connected, $M \sqcup_{\partial M \iso \partial N} N$ is connected.
\end{tcbthm}

\todo[inline]{add a graphical example. For reference Freed (page 10)}

\begin{remark}
While we do not give a formal proof of such result, we observe that the smooth structure on the topological pushout is not unique and relies on the choice of collars.
Our definition of bordisms with fixed collars is given precisely to specify this choice, making the gluing to be exactly the pushout in the proper category (which we can imagine as the category of ``manifolds with defined collars'').
However, different choices of collars give rise to diffeomorphic smooth structures. %This follows from a broader theorem we state below. \todo{find a reference. (sennò metti cock)}
\end{remark}

\begin{tcbthm}\label{th:collar_diffeo}
Let $\Sigma$ be an out-boundary of a bordism $M_0$ and an in-boundary of a bordism $M_1$ and consider $M_0 \sqcup_\Sigma M_1$ the pushout through $\Sigma$ of the two topological manifolds. 
Let $\alpha$, $\beta$ be two smooth structures on $M_0 \sqcup_\Sigma M_1$, which both induce the original smooth structures on $M_0$ and $M_1$ (via pullback along the inclusion maps).
Then there is a diffeomorphism $\phi \colon (M_0 \sqcup_\Sigma M_1, \alpha) \to (M_0 \sqcup_\Sigma M_1, \beta)$ such that its restricition on $\Sigma$ is the identity $\id_\Sigma$.
\end{tcbthm}



\section{A category of oriented boridsms}

Our goal is now to construct a category of oriented n-dimensional bordisms. 
The intuitive idea behind it is to take closed oriented $(n-1)$-dimensional manifolds as objects and oriented bordisms between them as morphisms.
The proper definition, however, requires some more refinement. 
We begin by adressing some techinical issues. %\todo{...ew, change tecnhicak issues (?)}
While equipping each bordism with an explicit choice of a collar allows us to properly define a gluing, this approach fails when trying to define a strict identity morphism.
Given a closed oriented $(n-1)$-dimensional manifold $\Sigma$, a candidate for $\id_\Sigma$ is given by the cylinder $\Sigma \times [0,1]$ with some choice of collars. 
However, gluing it to a bordism $M$ with one of the boundaries equal to $\Sigma$, only gives us a manifold \emph{diffeomorphic} to $M$. 
By considering the underlying topological spaces we easily understand that the only way to define a strict identity in this setting is to consider $\Sigma$ itself as a bordism, but this would not satisfy our definition which requires it to be an $n$-dimensional manifold.
Finally we recall how, when gluing manifolds, different choices of collars give rise to different but diffeomorphic smooth structures. 
This motivates the following framework: we define the bordism category by taking as morphisms bordisms ``up to diffeomorphism''.

\begin{tcbdfn}[Equivalent bordisms]
  Let $(M, i_0, i_1)$, $(M', i'_0, i'_1)$ be two oriented bordisms, both from $\Sigma_0$ to $\Sigma_1$. 
  We say $M$ and $M'$ are \emph{equivalent bordisms} if there exists an orientation preserving diffeomorphism $\psi \colon M \to M'$ making the following diagram commute.
  \[
  \begin{tikzcd}
    & M \arrow[dd, dashed, "\psi"] & \\
    \Sigma_0 \times \{0\} \iso \Sigma_0 \arrow[ur, "i_0"] \arrow[dr, "i'_0", swap] && \Sigma_1 \iso \Sigma_1 \times \{1\} \arrow[ul, "i_1", swap] \arrow[dl, "i'_1"] \\
    & M'&
  \end{tikzcd}
  \]
\end{tcbdfn}

This clearly defines an equivalence relation between bordisms. We can then consider the equivalence classes $[(M, i_0, i_1)]$.


\begin{remark}
In defining the equivalence class we are, in a way, forgetting the collar choice we gave in the definition of a bordism.
Indeed, when considering two bordisms $(M, i_0, i_1)$, $(M, j_0, j_1)$, we are asking for 
\[
i_0(\Sigma_0, 0) = (\partial M)_0 = j_0(\Sigma_0, 0) \qquad j_0(\Sigma_1, 1) = (\partial M)_1 = j_1(\Sigma_1, 1)
\]
hence making the diagram commute by just choosing the identity map on $M$.
From now on, when referring to an equivalence class of bordisms, we can forget the collars data and only consider the diffeomorphisms $\Sigma_0 \iso (\partial M)_0$ and $\Sigma_1 \iso (\partial M)_1$. 
We'll then just say that $M$ is a bordism from $\Sigma_0$ to $\Sigma_1$, without specifying further informations.
\end{remark}

\begin{tcblemma}[Composition of cobordism classes]
Given a bordism $M$ from $\Sigma_0$ to $\Sigma_1$ and a bordism $N$ from $\Sigma_1$ to $\Sigma_2$, we define their composition $MN$ from $\Sigma_0$ to $\Sigma_2$ as follows: we take any representative from each class, glue them and consider the equivalence class of the resulting gluing.
This composition is well defined.
\end{tcblemma}

\begin{proof}
Take the bordisms $(M, i_0, i_1)$, $(M', i'_0, i'_1)$ from the first equivalence class and $(N, i_0, i_1)$, $(N', i'_0, i'_1)$ from the second. This means we have two diffeomorphisms $\psi_M$, $\psi_N$ such that:
\[
\begin{tikzcd}
& M \arrow[dd, "\psi_0"] & & N \arrow[dd, "\psi_1"] & \\
\Sigma_0 \arrow[ur] \arrow[dr] & & \Sigma_{1} \arrow[ul] \arrow[ur] \arrow[dl] \arrow[dr] & & \Sigma_2 \arrow[ul] \arrow[dl] \\
%\Sigma_0 \times \{0\} \iso \Sigma_0 \arrow[ur] \arrow[dr] & & \Sigma_1 \times \{1\} \iso \Sigma_1 \iso \Sigma_{1} \times \{0\} \arrow[ul] \arrow[ur] \arrow[dl] \arrow[dr] & & \Sigma_2 \iso \Sigma_2 \times \{1\} \arrow[ul] \arrow[dl] \\
& M' & & N' & 
\end{tikzcd}
\]  
We can then consider the gluings $(MN, i_0, j_1)$ and $(M'N', i'_0, j'_1)$. 
By taking the pushout of the two diffeomorphisms in the category of continuous maps we get an homeomorphism $\psi \colon MN \to M'N'$. 
\todo[inline]{come si comporta? preserva la struttura liscia? Se lo fa sistema il finale}
Through such homeomorphism we can define a new smooth structure on $M'N'$, which by the previous theorem \ref{th:collar_diffeo} is diffeomorphic to the one induced by the gluing.
\end{proof}

\begin{tcbdfn}
The category of oriented n-dimensional bordisms $\bord{n}$ is defined as follows.
\begin{itemize}
  \item The objects are closed oriented $(n-1)$-dimensional manifolds
  \item For any $\Sigma_0, \Sigma_1 \in \obj{\bord{n}}$, morphisms are the equivalence classes of bordisms $M \colon \Sigma_0 \to \Sigma_1$
  \item Composition of morphisms is obtained by gluing
  \item For each object $\Sigma$, the identity map $\id_\Sigma$ is given by the bordism $\Sigma \times [0,1]$
\end{itemize}
\end{tcbdfn}

% def disjoint union of bordisms
\begin{dfnx}[Disjoint union of bordisms]
Given two bordisms $M \colon \Sigma_0 \to \Sigma_1$ and $N \colon \Sigma'_0 \to \Sigma'_1$, their disjoint union $M \disj N$ naturally defines a bordism from $\Sigma_0 \disj \Sigma'_0$ to $\Sigma_1 \disj \Sigma'_1$. 
This operation is precisely the coproduct in the category of smooth manifold, equipped with the unique orientation agreeing with the ones on $M$ and $N$.
\end{dfnx}

\begin{tcblemma} 
Let $\cat{C}$ be a category that admits coproducts. 
Then $(\cat{C}, \disj, 0)$, where $\disj$ denotes the coproduct and $0$ the initial object, is a monoidal category.
\end{tcblemma}

%\begin{proof} \todo[inline]{is this needed?}
%Firstly, let us prove the associativity axiom. Consider any objects $A$, $B$, $C$ in $\cat{C}$ and show that $(A \disj B) \disj C \iso A \disj (B \disj C)$.
%We hence take the objects $(A \disj B) \disj C$, $A \disj (B \disj C)$ and the corresponding inclusion maps. By the universal property of the coproduct $A \disj B$ we have a unique map $A \disj B \to A \disj (B \disj C)$ and
%\[
%by the universal property of the coproduct $(A \disj B) \disj C$, there exists a unique morphism $(A \disj B) \disj C \to A \disj (B \disj C)$.
%Similarly, by the universal property of $B \disj C$ and $A \disj (B \disj C)$, we have a unique morphism $A \disj (B \disj C) \to (A \disj B) \disj C$. By uniqueness of such maps we have the isomorphism we wished for.
%\[
%\begin{tikzcd}[column sep=huge, row sep=small]
%& A \disj (B \disj C) \arrow[dddd, bend left= 30, dashed] &\\
%A \arrow[ur, tail] \arrow[dd, tail] & & B \disj C \arrow[ul, tail] \arrow[dddl, dashed, "\exists !"] \\
%& B \arrow[ur, tail] \arrow[dl, tail] & \\
%A \disj B \arrow[dashed, uuur, "\exists !"] \arrow[dr, tail] & & C \arrow[dl, tail] \arrow[uu, tail] \\
%& (A \disj B) \disj C \arrow[uuuu, bend left= 30, dashed] & 
%\end{tikzcd}
%\]
%We now verify the unit axiom. %What we'll prove is that $A$ satisfies the universal property of the coproducts $A \disj 0$ and $0 \disj A$.
%Again in the diagram below we have that by the univ property of $0 \disj A$ there exists a unique arrow making the diagram commute. We can show such arrow is the inverse of the inclusion (again by univ property of $0 \disj A$)
%\[
%\begin{tikzcd}
%& 0 \disj A  \arrow[dd]& \\
%0 \arrow[ur] \arrow[dr] && A \arrow[ul] \arrow[dl, equal] \\
%& A &
%\end{tikzcd}
%\]
%Similarly, this holds when considering $A \disj 0$.
%Having defined these isomorphisms through the universal property we can be sure the coherence axioms are satisfied too.
%\end{proof}

\begin{tcbprp}
The category $\bord{n}$ has a monoidal structure given by the disjoint union of manifolds (and the initial object, being the empty manifold $\varnothing$).
%$(\bord{n}, \disj, \emptyset)$ is a monoidal category. (or better, is a monoidal structure on $\bord{n}$)
\end{tcbprp}

\begin{tcbprp}
Any diffeomorphism of $(n-1)$-dimensional manifolds $\Sigma_0$, $\Sigma_1$ define an equivalence class of (invertible) bordisms $M \colon \Sigma_0 \to \Sigma_1$.
\end{tcbprp}

\begin{dfnx}[The twist bordism]
The twist diffeomorphism of manifolds $\sigma \colon \Sigma \disj \Sigma' \to \Sigma' \disj \Sigma$ defines a cobordism in $\bord{n}$ which we'll denote as
$T_{\Sigma,\Sigma'} \colon \Sigma \disj \Sigma' \to \Sigma' \disj \Sigma$. 
\end{dfnx}

% maybe add T natural in Bord

\begin{tcbprp}
The category $\bord{n}$ has a symmetric monoidal structure $(\bord{n}, \disj, \varnothing, T)$
\end{tcbprp}

\section{Topological Quantum Field Theories}

\begin{tcbdfn} \label{def:tqft}
An \emph{$n$-dimensional topological quantum field theory} is a symmetric monoidal functor from $(\bord{n}, \disj, \varnothing, T)$ to $(\mathbf{Vect}, \tensor, \Bbbk, \sigma)$
\end{tcbdfn}

Historically, the first explicit axiomatization of TQFTs was given by Atiyah in 1988 \cite{Atiyah1988}. 
In the article we do not encounter the explicit notion of bordisms as we defined them. 
We here give a slightly refined version of the original axioms, stated in terms of bordisms. \todo{debbo citare che lho presa da cocco??}
%Let us see how they compare with the functorial definition we just stated.

\begin{tcbdfn}[Axiomatization of a TQFT]
An \emph{n-dimensional topological quantum field theory over a field $\Bbbk$} consists of:
\begin{itemize}
  \item a $\Bbbk$-vector space $Z(\Sigma)$ associated to each closed oriented $(n-1)$-dimensional manifold $\Sigma$
  \item a $\Bbbk$-linear map $Z(M) \colon Z(\Sigma_0) \to Z(\Sigma_1)$ associated to each $n$-dimensional bordism $M$ from $\Sigma_0$ to $\Sigma_1$
\end{itemize}
satisfying the following axioms:
\begin{enumerate}[{A}1]
\item \label{tqft:a1} Two equivalent bordisms $M \iso N$ have the same image through $Z$, namely $Z(M)=Z(N)$
\item \label{tqft:a2} The cylinder bordism $M = \Sigma \times I \colon \Sigma \to \Sigma$ is mapped to the identity
\[ Z(\Sigma \times I) = \id_{Z(\Sigma)} \colon Z(\Sigma) \to Z(\Sigma) \]
\item \label{tqft:a3} The gluing of two bordisms is mapped to the composition of their images, meaning that for two composable morphisms $M$, $N$, we have
\[ Z(MN) = Z(N) \circ Z(M) \]
\item \label{tqft:a4} The disjoint union of two bordisms is mapped to the tensor product of their images, meaning 
\[Z(M \disj N) = Z(M) \tensor Z(N) \]
\item \label{tqft:a5} The empty manifold $\varnothing$ is mapped to the ground field $\Bbbk$. It follows from A\ref{tqft:a2} that the empty bordism $\varnothing \times I$ is sent to the identity $\id_\Bbbk$.
\end{enumerate}
\end{tcbdfn}

\begin{remark}
It seems natural to interpret these axioms in the language of category theory.
Indeed, axiom A\ref{tqft:a1} well defines a map form $\bord{n}$ to $\mathbf{Vect}_\Bbbk$ and axioms A\ref{tqft:a2} and A\ref{tqft:a3} guarantee such map is really a functor.
Axioms A\ref{tqft:a4} and A\ref{tqft:a5} preserve the monoidal structure. To get a \emph{symmetric} monoidal functor, as defined in \ref{def:tqft}, we would have to add a sixth axiom:
\begin{enumerate}
  \item[(A6)] The twist bordism is mapped to the twist map of vector spaces, meaning 
  \[ Z(T_{M,N}) = \sigma_{Z(M),Z(N)} \] 
\end{enumerate}  
\end{remark}

\todo[inline]{we should also mention decomposition of bordism, snake lemma and dualizability somewhere}


