\chapter{Frobenius algebras}
\label{ch:frobenius}

\section{Four equivalent characterizations}
We begin by recalling the essential algebraic notions, which will be necessary to give a proper definition of a Frobenius algebra. 
Through the chapter, we let $\Bbbk$ denote a field.

\begin{tcbdfn}[Algebras over a field]
    A $\Bbbk$-algebra (i.e. an algebra over a field $\Bbbk$) is a $\Bbbk$-vector space $A$ equipped with two $\Bbbk$-linear maps 
    \[ \mu \colon A \tensor A \to \tensor A \qquad \eta \colon \Bbbk \to A  \]
    such that the following diagrams commute.
\[
\begin{tikzcd}
    & A \tensor A \tensor A \arrow[dl, "\mu \tensor \id_A", swap] \arrow[dr, "\id_A \tensor \mu"] & \\
    A \tensor A \arrow[dr, "\mu", swap] & & A \tensor A \arrow[dl, "\mu"] \\
    & A &
\end{tikzcd}
\]
\[
\begin{tikzcd}
    & A \tensor A \arrow[rd, "\mu"] & & A \tensor A \arrow[ld, "\mu", swap]& \\
    \Bbbk \tensor A \arrow[ru, "\eta \tensor \id_A"] \arrow[rr, "\pi", swap]& & A & & A \tensor \Bbbk \arrow[lu, "\id_A \tensor \eta", swap] \arrow[ll,"\pi"]  
\end{tikzcd}
\]
\end{tcbdfn}

This definition is equivalent to the standard notion of a (unital associative) $\Bbbk$-algebra from linear algebra: a $\Bbbk$-vector space equipped with an associative and unital bilinear product. 
Concretely, we can see that:
\begin{enumerate}
    \item \emph{Distributivity} follows from requiring $\mu$ to be $\Bbbk$-linear.
    \item \emph{Associativity} and \emph{compatibility with scalar product} follow from the two commutative diagrams.
    \item The \emph{unit element} is defined by the image of $1 \in \Bbbk$ through $\eta$.
\end{enumerate}


\begin{remark}
    It is no coincidence that these diagrams seem similar to the ones appearing at the beginning of our thesis in \ref{dfn:monoidal_category}. 
    $\Bbbk$-algebras are indeed internal monoids in the monoidal category $(\mathbf{Vect}_\Bbbk, \tensor, \Bbbk)$, while (strict) monoidal categories are monoids in $(\mathbf{Cat}, \times, \mathbf{1})$. 
    We'll define what this means in the next chapter. \todo{add ref to next chapter}
\end{remark}

\begin{tcbdfn}[Morphisms of $\Bbbk$-algebras]
    
\end{tcbdfn}

\begin{tcbdfn}[Linear forms]
\end{tcbdfn}

\begin{tcbdfn}[Duals]
\end{tcbdfn}

\begin{tcbdfn}[Pairings]
\end{tcbdfn}

\begin{tcbdfn}[Non degenerate pairings]
\end{tcbdfn}

\begin{tcblemma}
\end{tcblemma}

\section{Rigorous doodles} 
%% Alternative titolo: 
% "Drawing conclusions"?? (ma non sto concludendo nulla)
% "Reading between the lines"

Bla bla bla
