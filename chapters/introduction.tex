\dropchapter{2in}
\chapter*{Introduction}
\addcontentsline{toc}{chapter}{Introduction}

% \epigraph{In Egypt, geometry was created to measure the land. Similar motivations, on a somewhat larger scale, led Gauss to the intrinsic differential geometry of surfaces in space. Newton created the calculus to study the motion of physical objects (apples, planets, etc.) and Poincaré was similarly impelled toward his deep and far-reaching topological view of dynamical systems. This symbiosis between mathematics and the study of the physical universe, which nourished both for thousands of years, began to weaken, however, in the early years of the last century. Mathematics was increasingly taken with the power of abstraction and physicists had no time to pursue charming generalizations in the hope that the path might lead somewhere. And so, the two parted company. Nature, however, disapproved of the divorce and periodically arranged for the disaffected parties to be brought together once again.}{G. L. Naber, \textit{Topology, Geometry and Gauge Fields}}

\epigraphhead[70]{\epigraph{There are many reasons to study topological quantum field theories, but one reason is that they exhibit a beautiful relationship between algebra and geometry}{Christopher John Schommer-Pries, \textit{The Classification of Two-Dimensional Extended Topological Field Theories}}}
\undodrop


First we reassure the reader: this is a thesis in pure mathematics. Its goal is to develop the generator-and-relations presentation of a category that (the author has heard) plays an important role in modern theoretical physics research, and to use it to establish a link between geometry (i.e., the representations of the category of $ 2 $-dimensional oriented bordisms) and algebra (i.e., some less known kind of algebraic structures of increasing popularity among logicians and computer scientists that go under the name of ``Frobenius algebras''). After reviewing some basic vocabulary from category theory in~Chapter \ref{chap:cats}, we introduce oriented bordisms and their ambient category $ \mathbf{Bord}(2) $ in Chapter~\ref{chap:bord}, along with the \emph{linear representations} of this category, i.e., the symmetric monoidal functors from $ \mathbf{Bord}(2) $ to the (symmetric monoidal) category of (finite-dimensional) vector spaces over an arbitrary field.

\todo[inline, color=yellow]{In laywoman's terms: the classification of $ (n - 1) $-dimensional manifolds up to isomorphism is a difficult task. Why not to relax the comparison relation to one which is easier to deal with (and at the same time not vacuous)? Puoi continuare copiando due cazzate along these lines da Hirsch. IO vado  a nanna <3. P.S. Ho aggiunto una direttiva includeonly nel main}
\lipsum
%%% Local Variables:
%%% mode: LaTeX
%%% TeX-master: "../main"
%%% End:
