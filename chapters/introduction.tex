\dropchapter{2in}
\chapter*{Introduction}
\addcontentsline{toc}{chapter}{Introduction}

% \epigraph{In Egypt, geometry was created to measure the land. Similar motivations, on a somewhat larger scale, led Gauss to the intrinsic differential geometry of surfaces in space. Newton created the calculus to study the motion of physical objects (apples, planets, etc.) and Poincaré was similarly impelled toward his deep and far-reaching topological view of dynamical systems. This symbiosis between mathematics and the study of the physical universe, which nourished both for thousands of years, began to weaken, however, in the early years of the last century. Mathematics was increasingly taken with the power of abstraction and physicists had no time to pursue charming generalizations in the hope that the path might lead somewhere. And so, the two parted company. Nature, however, disapproved of the divorce and periodically arranged for the disaffected parties to be brought together once again.}{G. L. Naber, \textit{Topology, Geometry and Gauge Fields}}

\epigraphhead[70]{\epigraph{There are many reasons to study topological quantum field theories, but one reason is that they exhibit a beautiful relationship between algebra and geometry}{Christopher John Schommer-Pries, \textit{The Classification of Two-Dimensional Extended Topological Field Theories}}}
\undodrop

First we reassure the reader: this is a thesis in pure mathematics. Its goal is to develop the generator-and-relations presentation of a category that (the author has heard) plays an important role in modern theoretical physics research, and to use it to establish a link between geometry (i.e., the representations of the category of $ 2 $-dimensional oriented bordisms) and algebra (i.e., some less known kind of algebraic structures of increasing popularity among logicians and computer scientists that go under the name of ``Frobenius algebras'').

After reviewing some basic vocabulary from category theory in Chapter~\ref{ch:cats}, we introduce oriented bordisms and their ambient category $ \mathbf{Bord}(n) $ in Chapter~\ref{ch:bord} along with the \emph{linear representations} of this category, i.e., the symmetric monoidal functors from $ \mathbf{Bord}(n) $ to the (symmetric monoidal) category of (finite-dimensional) vector spaces over an arbitrary field. Topological field theories arose in the eighties as an attempt to formalize some constructions appearing in quantum field theory. A TQFT assigns to each $(n-1)$-dimensional ``geometric object'' a vector space of ``states'', and to each ``worldvolume'' traced by a geometric object an ``evolution operator'' relating its initial and final states. The ``geometric objects'' we have in mind are $(n-1)$-dimensional manifolds and the ``worldvolumes'' they trace occur here as $n$-dimensional bordisms between them. While this intuitive picture alone legitimates a genuine interest in studying monoidal functors from categories of bordisms to categories of vector spaces, we will see that the rich geometric structure of the bordism category endows the category of its linear representation (i.e., a category of TQFTs) with nice algebraic properties. In Chapter~\ref{ch:bord2} we introduce the notion of generators and relations for a monoidal category, and, relying on the classification theorem for $2$-dimensional surfaces, we make the geometric structure alluded above explicit, presenting the $2$-dimensional bordism category concretely in terms of generators and relations. From the appropriate perspective, the relations presenting this bordism category look exactly like the axioms that define a Frobenius algebra. Chapter~\ref{ch:frobenius} develops a powerful graphical calculus for dealing with Frobenius algebras (and more generally, with algebraic structures defined within monoidal categories in general). We employ this calculus to analyze the multiple equivalent definitions of a Frobenius algebra, which can be described in so many words just as a monoid object in the category of vector spaces which is also a comonoid object in a compatible way. With these premises in place we finally reach the main theorem of this thesis in Chapter~\ref{ch:grand-finale}, which basically states that defining a $2$-dimensional TQFT is the same as choosing a commutative Frobenius algebra. We also see how this equivalence result is just an instance of a broader theorem: \bord{2} is the free symmetric monodial category over an internal Frobenius object.

The idea of writing a thesis on this topic arose from a wish to approach monoidal categories and their internal objects through a concrete example. The narrow time constraints determined the scope of this exposition, which offers nothing more than a quick informal route towards the equivalence theorem between $2$-dimensional TQFTs and Frobenius algebras. The serious reader is encouraged to turn (for instance) to~\cite{TuraevVirelizieMonoidal} (for the general theory), \cite{moore2025tasilecturestopologicalfield} (for a readable introduction to the physical side of the story), \cite{freed2013bordism} (for bordism theory), \cite{Atiyah1988} (for an original source) and of course to~\cite{kock2003frobenius}. The majority of this work clearly relies on \cite{kock2003frobenius} and can largely be seen as a retelling of its tale of \guillemotleft the commutative Frobenius and the princess $\mathbf{Bord}(2)$\guillemotright, even if by a less skillful bard.

\begin{flushright}
  Padova, December 2025
\end{flushright}

% defining free monoidal categories over some particular objects.

% our final step is 


% the final step needed to define an explicit link between bordisms and Frobenius algebras are representations of said bordisms. Indeed in Chapter~\ref{ch:grand-finale} we finally state that defining a 2-dimensional TQFT is the same as choosing a commutative Frobenius algebra (and viceversa).
% Those "Frobenius algebra-looking" relations come to life as Frobenius algebras when evaluated on the vector spaces. 

% prendono vita come algebre di frobenius quando le valutiamo su 

% This last notion

% This last notion is given for a reason: it is equivalent to the one of TQFTs.

 

% Things become even more concrete and structured in Chapter~\ref{ch:bord2} where we consider the particular case of dimension 2. Indeed, having an already established classification of topological surfaces allows us to give a presentation of the category \bord{2} in terms of generators and relations and ask ourselves what a representation of such category looks like. At the end we will discover this defines exactly a structure of Frobenius algebra. In Chapter~\ref{ch:frobenius} we'll have a close encounter with these algebras, through the help of a graphical language. 

% In laywoman's terms: the classification of $ (n - 1) $-dimensional manifolds up to isomorphism is a difficult task. Why not to relax the comparison relation to one which is easier to deal with (and at the same time not vacuous)?

% When trying to partition $(n-1)$-dimensional manifold into equivalence classes, instead of asking for two manifolds to be diffeomorphic, we ask if they together bound some $n$-dimensional manifold. This relation is broader than the one of being diffeomorphic (diffeomorphic manifolds are bordant), but way more practical. In this way we can forget the tedious details, while still remaining (deeply?) connected to topology and physics (and algebra). 


%%% Local Variables:
%%% mode: LaTeX
%%% TeX-master: "../main"
%%% End:

%% ASTRATTO
%% This thesis explores the equivalence between two dimensional topological quantum field theories and Frobenius algebras. We first provide a complete presentation by generators and relations of the $2$-dimensional oriented bordism category as a symmetric monoidal category. This presentation is then used to classify the $2$-dimensional topological field theories with values in an arbitrary target symmetric monoidal category.
 
