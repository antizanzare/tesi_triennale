\dropchapter{2in}
\chapter*{Introduction}
\addcontentsline{toc}{chapter}{Introduction}

% \epigraph{In Egypt, geometry was created to measure the land. Similar motivations, on a somewhat larger scale, led Gauss to the intrinsic differential geometry of surfaces in space. Newton created the calculus to study the motion of physical objects (apples, planets, etc.) and Poincaré was similarly impelled toward his deep and far-reaching topological view of dynamical systems. This symbiosis between mathematics and the study of the physical universe, which nourished both for thousands of years, began to weaken, however, in the early years of the last century. Mathematics was increasingly taken with the power of abstraction and physicists had no time to pursue charming generalizations in the hope that the path might lead somewhere. And so, the two parted company. Nature, however, disapproved of the divorce and periodically arranged for the disaffected parties to be brought together once again.}{G. L. Naber, \textit{Topology, Geometry and Gauge Fields}}

\epigraphhead[70]{\epigraph{There are many reasons to study topological quantum field theories, but one reason is that they exhibit a beautiful relationship between algebra and geometry}{Christopher John Schommer-Pries, \textit{The Classification of Two-Dimensional Extended Topological Field Theories}}}
\undodrop


First we reassure the reader: this is a thesis in pure mathematics. Its goal is to develop the generator-and-relations presentation of a category that (the author has heard) plays an important role in modern theoretical physics research, and to use it to establish a link between geometry (i.e., the representations of the category of $ 2 $-dimensional oriented bordisms) and algebra (i.e., some less known kind of algebraic structures of increasing popularity among logicians and computer scientists that go under the name of ``Frobenius algebras''). After reviewing some basic vocabulary from category theory in Chapter~\ref{ch:cats}, we introduce oriented bordisms and their ambient category $ \mathbf{Bord}(n) $ in Chapter~\ref{ch:bord}, along with the \emph{linear representations} of this category, i.e., the symmetric monoidal functors from $ \mathbf{Bord}(n) $ to the (symmetric monoidal) category of (finite-dimensional) vector spaces over an arbitrary field.

In laywoman's terms: the classification of $ (n - 1) $-dimensional manifolds up to isomorphism is a difficult task. Why not to relax the comparison relation to one which is easier to deal with (and at the same time not vacuous)?

When trying to partition $(n-1)$-dimensional manifold into equivalence classes, instead of asking for two manifolds to be diffeomorphic, we ask if they together bound some $n$-dimensional manifold. This relation is broader than the one of being diffeomorphic (diffeomorphic manifolds are bordant), but way more practical. In this way we can forget the tedious details, while still remaining (deeply?) connected to topology and physics (and algebra). Things become even more concrete and structured (?) in Chapter~\ref{ch:bord2} where we consider the particular case of dimension 2. Indeed, having an already established classification of topological surfaces allows us to give a presentation of the category \bord{2} in terms of generators and relations and ask ourselves what a representation of such category looks like. At the end we will discover this defines exactly a structure of Frobenius algebra. In Chapter~\ref{ch:frobenius} we'll have a close encounter with these algebras, through the help of a graphical language. With these premises in place we finally reach the main equivalence in Chapter~\ref{ch:grand-finale} which basically states that defining a 2dimensional TQFT is the same as choosing a commutative Frobenius algebra (and viceversa). We also see how this is just the an instance of a broader concept: defining free monoidal categories over some particular objects. 
 

The idea of writing a thesis on this topic arose from a (desire???) to approach monoidal categories and their internal objects through a concrete example. During the work, I (??) got lost many times when trying to expand on other concepts. This is also due to the fact that the right setting as of nowadays seems to be the one of higher categories and extended bordisms. Due to the limited amount of time, I could only take a look at this more advanced topics, which could not make it in the thesis. 

The majority of this work is clearly based on \cite{kock2003frobenius} and can largely be seen as a retelling of its tale of \emph{the commutative Frobenius and the princess $\mathbf{Bord}(2)$}, even if by a less skillful bard. 

%%% Local Variables:
%%% mode: LaTeX
%%% TeX-master: "../main"
%%% End:
