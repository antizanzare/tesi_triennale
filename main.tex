
\documentclass[a4paper,11pt]{book}


% lingue
\usepackage[T1]{fontenc}
\usepackage[utf8]{inputenc}

\usepackage[italian, english]{babel}
\usepackage[babel=true]{microtype}

%\usepackage{textcomp}

% Margini pari
\usepackage[a4paper
,top=4.5cm
,bottom=4cm
,inner=3.5cm
,outer=3.5cm]
{geometry}



% Margini sfasati per stampa a libro
% \usepackage[a4paper
% ,top=4.5cm
% ,bottom=4cm
% ,inner=4.25cm
% ,outer=2.75cm
% ]{geometry}

% ams-cose
\usepackage{amsmath, amsfonts, amssymb, amsthm}
\usepackage{mathtools, nccmath}
\usepackage{mathrsfs} % per \mathscr
% \usepackage{bbold}

%\usepackage{tocbibind} % Per avere l'entrata "Indice" nell'indice, commentare la riga per rimuoverlo

% pacchetti da template
\usepackage{graphicx} % Pacchetto per immagini

% Per produrre un pdf/a, de-commentare la prossima riga
%\usepackage[a-1b]{pdfx}

% miscellanea
\usepackage{lipsum}
\usepackage{enumerate}
\usepackage{etaremune} % reverse enumerate
\usepackage{multirow} % multirow tab cells
%\usepackage{ebproof} % inference trees
\usepackage{epigraph}  % quotations at the start (or end) of a chapter
\usepackage{todonotes}
\newcommand{\fouche}[1]{\todo[inline, caption=0,color=red!10,bordercolor=red!10]{\textbf{fouche said}: #1}}
\setuptodonotes{color=blue!30}
\usepackage{wasysym} % some symbols such as \sqsubset, \sqsupset, \lightning , \paragraph, notes, polygons
%\usepackage{xfrac} % split levels fractions

% per i disegnetti
\usepackage{float} % containers for things in a document that cannot be broken over a page (es. figure)
\usepackage{pgfplots} % plots graphs ?
\pgfplotsset{compat=1.12}
\usepgfplotslibrary{fillbetween}
\usepackage{subfig} % more than one image in a figure
\usepackage{xcolor}


% NOTE: this is "Color Blind Safe Palette of IBM Design Language v1 Color"
\definecolor{ibmBlue}{HTML}{648fff}
\definecolor{ibmPurple}{HTML}{785ef0}
\definecolor{ibmMagenta}{HTML}{dc267f}
\definecolor{ibmOrange}{HTML}{fe6100}
\definecolor{ibmYellow}{HTML}{ffb000}
%addition
\definecolor{ibmGreen}{HTML}{008000}

%tikz
\usepackage{tikz}
\usetikzlibrary{babel}
\usetikzlibrary{cd} % tikz-cd
\usetikzlibrary{graphs} % for simple graphs
\usetikzlibrary{calc} % functions in tikz
\usetikzlibrary{positioning} % position relative to other nodes
\usetikzlibrary{shapes.geometric, shapes.misc}
\usetikzlibrary{decorations.markings, arrows, patterns}
\usetikzlibrary{matrix} % matrices as tikzpictures
\usetikzlibrary{backgrounds}
\usetikzlibrary{fit} % ?? fitting a node so that it contains a set of coordinates.





% Bibliografia da template
\usepackage[backend=biber, style=alphabetic]{biblatex}
\usepackage{csquotes}
\addbibresource{references.bib}

%% Libreria per indice
\usepackage{imakeidx}
\makeindex




%%%%% girly environments %%%%
\usepackage{tcolorbox}
\tcbuselibrary{breakable}
\tcbuselibrary{theorems}
\tcbuselibrary{skins}
\usepackage{etoolbox}




% THEOREMS ENV
\tcbset{genericstyle/.style=
		{enhanced, 
		breakable, 
		theorem style=plain, %to invert th and number use theoremstyle=change
		colback=#1!5,
		colframe=#1!70!black,
		fonttitle=\bfseries,
		coltitle=#1!60!black,
		frame hidden,
		arc=0mm,left=0mm, right=0mm, top=0mm, bottom=0mm, boxsep=1mm, left skip=-1mm, right skip=-1mm
		}	
	}

\newcounter{main_counter}[chapter]

% Plain thm environment
\newcommand{\createtheoremenv}[4]{%
    % Numbered version
    \newtcolorbox[use counter=main_counter, number within=chapter]{#1}[1][]{
        genericstyle={#4},
        fonttitle=\bfseries\upshape,
        fontupper=\itshape,
        title=#2~\thetcbcounter\ifstrempty{##1}{}{ ##1},
        #3
    }%
    % Unnumbered version  
    \newtcolorbox{#1-}[1][]{
        genericstyle={#4},
        fonttitle=\bfseries\upshape,
        fontupper=\itshape,
        title=#2\ifstrempty{##1}{}{: ##1},
        #3
    }%
}


% create all the theorem environments
\createtheoremenv{tcbprp}{Proposition}{}{ibmBlue}
\createtheoremenv{tcblemma}{Lemma}{}{ibmBlue}
\createtheoremenv{tcbthm}{Theorem}{}{ibmOrange}
\createtheoremenv{tcbcorollary}{Corollary}{}{green}
%\createtheoremenv{tcbteo}{Teorema}{}{green}
%\createtheoremenv{tcbcorollario}{Corollario}{}{green}
%\createtheoremenv{tcbitaprp}{Proposizione}{}{green}



% Definition style environments (no italics)
\newcommand{\createdefenv}[4]{%
    % Numbered version
    \newtcolorbox[use counter=main_counter, number within=chapter]{#1}[1][]{
        genericstyle={#4},
        title=#2~\thetcbcounter\ifstrempty{##1}{}{ ##1},
        #3
    }%
    % Unnumbered version
    \newtcolorbox{#1-}[1][]{
        genericstyle={#4},
        title=#2\ifstrempty{##1}{}{: ##1},
        #3
    }%
}

% create all the def environments
\createdefenv{tcbdfn}{Definition}{}{ibmGreen}
\createdefenv{tcbdfnx}{}{}{ibmGreen}
\createdefenv{exercise}{Exercise}{}{green}
%\createdefenv{tcbitadfn}{Definizione}{}{ibmGreen}
%\createdefenv{tcbesercizio}{Esercizio}{}{green}


% Observation/Remark style environments (italic title, no italics content)
\newcommand{\createremarkenv}[4]{%
    % Numbered version
    \newtcolorbox[use counter=main_counter, number within=chapter]{#1}[1][]{
        genericstyle={#4},
        before skip=1pt,
        after skip=1pt,
        fonttitle=\itshape,
        title=#2~\thetcbcounter\ifstrempty{##1}{}{ ##1},
        #3
    }%
    % Unnumbered version
    \newtcolorbox{#1-}[1][]{
        genericstyle={#4},
        before skip=1pt,
        after skip=1pt,
        fonttitle=\itshape,
        title=#2\ifstrempty{##1}{}{: ##1},
        #3
    }%
}

%create all remark environments
\createremarkenv{tcbosservazione}{Osservazione}{}{blue}
\createremarkenv{tcbremark}{Remark}{}{blue}
\createremarkenv{tcbesempio}{Esempio}{}{blue}
\createremarkenv{tcbexample}{Example}{}{ibmBlue}
\createremarkenv{tcbdomanda}{Domanda}{}{blue}




%%plain
%\newtcbtheorem[number within=section, use counter=main_counter]{tcbprp}{Proposizione}{genericstyle={green},fonttitle=\bfseries\upshape,fontupper=\itshape}{prp}
%
%\newtcbtheorem{tcbprp-}{Proposizione}{genericstyle={green},fonttitle=\bfseries\upshape,fontupper=\itshape}{prpno}
%
%\newtcbtheorem[number within=section, use counter=main_counter]{tcbengprp}{Proposition}{genericstyle={green},fonttitle=\bfseries\upshape,fontupper=\itshape}{eprp}
%
%\newtcbtheorem{tcbengprp-}{Proposition}{genericstyle={green},fonttitle=\bfseries\upshape,fontupper=\itshape}{eprpno}
%
%\newtcbtheorem[number within=section, use counter=main_counter]{tcblemma}{Lemma}{genericstyle={green},fonttitle=\bfseries\upshape,fontupper=\itshape}{lem}
%
%\newtcbtheorem{tcblemma-}{Lemma}{genericstyle={green},fonttitle=\bfseries\upshape,fontupper=\itshape}{lemno}
%
%\newtcbtheorem[number within=section, use counter=main_counter]{tcbteo}{Teorema}{genericstyle={green},fonttitle=\bfseries\upshape,fontupper=\itshape}{teo}
%
%\newtcbtheorem{tcbteo-}{Teorema}{genericstyle={green},fonttitle=\bfseries\upshape,fontupper=\itshape}{teono}
%
%\newtcbtheorem[number within=section, use counter=main_counter]{tcbthm}{Theorem}{genericstyle={ibmOrange},fonttitle=\bfseries\upshape,fontupper=\itshape}{thm}
%
%\newtcbtheorem{tcbthm-}{Theorem}{genericstyle={ibmOrange},fonttitle=\bfseries\upshape,fontupper=\itshape}{thmno}
%
%\newtcbtheorem[number within=section, use counter=main_counter]{tcbcorollario}{Corollario}{genericstyle={green},fonttitle=\bfseries\upshape,fontupper=\itshape}{cor}
%
%\newtcbtheorem{tcbcorollario-}{Corollario}{genericstyle={green},fonttitle=\bfseries\upshape,fontupper=\itshape}{corno}
%
%\newtcbtheorem[number within=section, use counter=main_counter]{tcbcorollary}{Corollary}{genericstyle={green},fonttitle=\bfseries\upshape,fontupper=\itshape}{ecor}
%
%\newtcbtheorem{tcbcorollary-}{Corollary}{genericstyle={green},fonttitle=\bfseries\upshape,fontupper=\itshape}{ecorno}
%
%
%%definition
%\newtcbtheorem[number within=section, use counter=main_counter]{tcbdfn}{Definizione}{genericstyle={ibmGreen}}{dfn}
%
%\newtcbtheorem{tcbdfn-}{Definizione}{genericstyle={green}}{dfnno}
%
%\newtcbtheorem[number within=section, use counter=main_counter]{tcbengdfn}{Definition}{genericstyle={ibmGreen}}{edfn}
%
%\newtcbtheorem{tcbengdfn-}{Definition}{genericstyle={green}}{edfnno}
%
%\newtcbtheorem[number within=section, use counter=main_counter]{tcbesercizio}{Esercizio}{genericstyle={green}}{es}
%
%\newtcbtheorem{tcbesercizio-}{Esercizio}{genericstyle={green}}{esno}
%
%\newtcbtheorem[number within=section, use counter=main_counter]{exercise}{Exercise}{genericstyle={green}}{enges}
%
%\newtcbtheorem{exercise-}{Exercise}{genericstyle={green}}{engesno}
%
%%definition
%\newtcbtheorem[number within=section, use counter=main_counter]{tcbosservazione}{Osservazione}{genericstyle={blue}, before skip=1pt,after skip=1pt, fonttitle=\itshape}{oss}
%
%\newtcbtheorem{tcbosservazione-}{Osservazione}{genericstyle={blue}, before skip=1pt,after skip=1pt, fonttitle=\itshape}{ossno}
%
%\newtcbtheorem[number within=section, use counter=main_counter]{tcbremark}{Remark}{genericstyle={blue}, before skip=1pt,after skip=1pt, fonttitle=\itshape}{rem}
%
%\newtcbtheorem{tcbremark-}{Remark}{genericstyle={blue}, before skip=1pt,after skip=1pt, fonttitle=\itshape}{remno}
%
%\newtcbtheorem[number within=section, use counter=main_counter]{tcbesempio}{Esempio}{genericstyle={blue}, before skip=1pt,after skip=1pt, fonttitle=\itshape}{ex}
%
%\newtcbtheorem{tcbesempio-}{Esempio}{genericstyle={blue}, before skip=1pt,after skip=1pt, fonttitle=\itshape}{exno}
%
%\newtcbtheorem[number within=section, use counter=main_counter]{tcbexample}{Example}{genericstyle={blue}, before skip=1pt,after skip=1pt, fonttitle=\itshape}{engex}
%
%\newtcbtheorem{tcbexample-}{Example}{genericstyle={blue}, before skip=1pt,after skip=1pt, fonttitle=\itshape}{engexno}
%
%\newtcbtheorem[number within=section, use counter=main_counter]{tcbdomanda}{Domanda}{genericstyle={blue}, before skip=1pt,after skip=1pt, fonttitle=\itshape}{dom}
%
%\newtcbtheorem{tcbdomanda-}{Domanda}{genericstyle={blue}, before skip=1pt,after skip=1pt, fonttitle=\itshape}{domno}


%%%% classic environments %%%%
\usepackage{theoremref}
%% \swapnumbers
%% Definizione di ambienti Teorema
%
\numberwithin{equation}{chapter}
%% Numerazione in capitolo
%
%% Ambienti in italico
\theoremstyle{plain}
%\newtheorem{thm}{Theorem}[chapter] % Se si vuole la numerazione in sezione, (e.g, Teorema 1.1.1), cambiare "chapter" con "section"
%\newtheorem*{thm*}{Theorem}
%
\newtheorem{prp}[main_counter]{Proposition}
\newtheorem*{prp*}{Proposition}
%
%\newtheorem{coroll}[prp]{Corollary}
%\newtheorem*{coroll*}{Corollary}
%
%\newtheorem{lemma}[prp]{Lemma}
%\newtheorem*{lemma*}{Lemma}
%
%
%% Ambienti non in italico
\def\th@definition{%
  \thm@notefont{}% same as heading font
  \normalfont % body font
}
\theoremstyle{definition}
%\newtheorem*{dfn*}{Definition}
%
\newtheorem{dfnx}[main_counter]{}
%
%
%% Ambienti in stile "Remark"
\theoremstyle{remark}
\newtheorem{remark}[main_counter]{Remark}
\newtheorem*{remark*}{Remark}
%
\newtheorem{example}[main_counter]{Example}
\newtheorem*{example*}{Example}
%

%\newtheorem{note}[prp]{Note}
%\newtheorem*{note*}{Note}
%
%%\newtheorem{domanda}[prp]{Domanda}
%%\newtheorem*{domanda*}{Domanda}
%%\newtheorem{soluzione}[prp]{Soluzione}
%%\newtheorem*{soluzione*}{Soluzione}
%%\newtheorem{notazione}[prp]{Notazione}
%%\newtheorem{assioma}[prp]{Assioma}


\newenvironment{myproof}{\proof\mbox{}\\*}{\endproof}


% definizioni
\newcommand{\emdash}{\text{-}}
\newcommand{\symmdiff}{\mathbin{\scriptstyle\bigtriangleup}}
\newcommand{\Sub}{\mathrm{Sub}}
\newcommand{\Hom}{\mathrm{Hom}}
\newcommand{\mono}{\rightarrowtail}
\newcommand{\epi}{\twoheadrightarrow}
\newcommand{\varx}{\mathrm{x}}		
\DeclarePairedDelimiter{\abs}{\lvert}{\rvert}
\DeclarePairedDelimiter{\norm}{\lVert}{\rVert}
\DeclarePairedDelimiter\ceil{\lceil}{\rceil}
\DeclarePairedDelimiter\floor{\lfloor}{\rfloor}
% Furto da itaca.sty
\newcommand{\cat}[1]{\mathcal{#1}}
\newcommand{\nat}{\Rightarrow}
\newcommand{\fun}{\longrightarrow}
\newcommand{\const}{\text{c}}
%\newcommand{\implies}{\Rightarrow}
\newcommand{\whisk}{innesto\@\xspace}
\newcommand{\zzg}{zig-zag\@\xspace}

\newcommand{\xto}[1]{\xrightarrow{#1}}
\newcommand{\xot}[1]{\xleftarrow{#1}}
\newcommand{\xfun}[1]{\xlongrightarrow{#1}}
\newcommand{\xnat}[1]{\xRightarrow{#1}}

\newcommand{\genArrow}{\boxed{\to}}

\newcommand{\singleton}{\{\bullet\}}

\newcommand{\iso}{\cong}
\newcommand{\eqv}{\simeq}
\newcommand{\defeq}{\coloneqq}


% smaller bullets
\renewcommand\labelitemi{$\vcenter{\hbox{\tiny$\bullet$}}$}

\newcommand{\sq}{\mathbin\square}
\newcommand{\obj}[1]{\mathrm{Obj}({#1})}
\newcommand{\arr}[1]{\mathrm{Arr}({#1})}
\newcommand{\id}{\mathrm{id}}

\newcommand{\moncat}{\textbf{MonCat}}
\newcommand{\brmoncat}{\textbf{BrMonCat}}
\newcommand{\symmoncat}{\textbf{SymMonCat}}
\newcommand{\bord}[1]{\textbf{Bord}({#1})}
\newcommand{\disj}{\amalg}
\newcommand{\tensor}{\otimes}


\newcommand{\varsq}{\mathbin \boxtimes}
\usepackage[colorlinks=true, allcolors=black]{hyperref}


\usepackage{tikz}
\usetikzlibrary{shadows, patterns, tqft}



% ============================= BEGIN DOCUMENT ===========================================
\begin{document}

%\pagenumbering{roman}

\frontmatter

% Copertina in bianco e nero per la stampa su pelle (commentare le prossime due righe se non serve)
%
\newgeometry{left=3cm,right=3cm,bottom=3cm,top=3cm}
\begin{titlepage}


 \centering
 \begin{figure}[h]
  \includegraphics[height=3cm]{Unipd_logo_black.pdf}
  \hfill
  \includegraphics[height=3cm]{DM_logo_black.pdf}
 \end{figure}

 \vspace{15pt}

 \hrule
 \vspace{10pt}
 \Large{\scshape Università degli studi di Padova}\\
 \Large{\scshape Dipartimento di Matematica \textquotedblleft Tullio Levi-Civita\textquotedblright}\\
 \Large {\scshape Corso di laurea in Matematica}
 \vspace{10pt}
 \hrule


 \vfill


 \Huge{\bfseries Titolo della tesi} \\
 \vspace{5pt}
 \Large{\bfseries Sottotitolo della tesi} \\


 \vfill

 \hrule
 \vspace{35pt}
 \begin{minipage}{0.48\textwidth}
  \begin{flushleft} \large
   \emph{Relatore:} \\
   Prof. Nome Cognome \\ % Supervisor's Name
   ~ \\
  \end{flushleft}
 \end{minipage}
 \begin{minipage}{0.48\textwidth}
  \begin{flushright} \large
   \emph{Candidato:}\\
   Chiara Poles\\
   Matricola 2011061
  \end{flushright}
 \end{minipage} \\
 \vspace{25pt}
 \large{Anno accademico 2024/2025 - 12.12.2025} % Inserire anno accademico e giorno di laurea

\end{titlepage}
\restoregeometry





%\thispagestyle{empty}
%\cleardoublepage

% Copertina a colori
%

\newgeometry{left=3cm,right=3cm,bottom=3cm,top=3cm}
\begin{titlepage}


 \centering
 \begin{figure}[h]
  \includegraphics[height=3cm]{../images/cover/Unipd_logo.pdf}
  \hfill
  \includegraphics[height=3cm]{../images/cover/DM_logo.pdf}
 \end{figure}

 \vspace{15pt}

 \hrule
 \vspace{10pt}
 \Large{\scshape Università degli studi di Padova}\\
 \Large{\scshape Dipartimento di Matematica \textquotedblleft Tullio Levi-Civita\textquotedblright}\\
 \Large {\scshape Corso di laurea in Matematica}
 \vspace{10pt}
 \hrule


 \vfill


 \Huge{\bfseries An Introduction to Frobenius Algebras and 2D TQFTs } \\
 %\vspace{5pt}
 %\Large{\bfseries}\\


 \vfill

 \hrule
 \vspace{35pt}
 \begin{minipage}{0.48\textwidth}
  \begin{flushleft} \large
   \emph{Relatore:} \\
   Prof. Samuele Maschio \\
   \emph{Correlatore:} \\
   Prof. Fosco Loregian \\
   ~ \\
  \end{flushleft}
 \end{minipage}
 \begin{minipage}{0.48\textwidth}
  \begin{flushright} \large
   \emph{Candidato:}\\
   Chiara Poles\\
   Matricola 2011061
  \end{flushright}
 \end{minipage} \\
 \vspace{25pt}
 \large{Anno accademico 2024/2025 - 12.12.2025} % Inserire anno accademico e giorno di laurea

\end{titlepage}
\restoregeometry



%\thispagestyle{empty}
%\cleardoublepage



% Pagina per la dedica e/o citazioni


% Index
\tableofcontents

\mainmatter

% Introduction
%\dropchapter{2in}
\chapter*{Introduction}
\addcontentsline{toc}{chapter}{Introduction}

% \epigraph{In Egypt, geometry was created to measure the land. Similar motivations, on a somewhat larger scale, led Gauss to the intrinsic differential geometry of surfaces in space. Newton created the calculus to study the motion of physical objects (apples, planets, etc.) and Poincaré was similarly impelled toward his deep and far-reaching topological view of dynamical systems. This symbiosis between mathematics and the study of the physical universe, which nourished both for thousands of years, began to weaken, however, in the early years of the last century. Mathematics was increasingly taken with the power of abstraction and physicists had no time to pursue charming generalizations in the hope that the path might lead somewhere. And so, the two parted company. Nature, however, disapproved of the divorce and periodically arranged for the disaffected parties to be brought together once again.}{G. L. Naber, \textit{Topology, Geometry and Gauge Fields}}

\epigraphhead[70]{\epigraph{There are many reasons to study topological quantum field theories, but one reason is that they exhibit a beautiful relationship between algebra and geometry}{Christopher John Schommer-Pries, \textit{The Classification of Two-Dimensional Extended Topological Field Theories}}}
\undodrop

First we reassure the reader: this is a thesis in pure mathematics. Its goal is to develop the generator-and-relations presentation of a category that (the author has heard) plays an important role in modern theoretical physics research, and to use it to establish a link between geometry (i.e., the representations of the category of $ 2 $-dimensional oriented bordisms) and algebra (i.e., some less known kind of algebraic structures of increasing popularity among logicians and computer scientists that go under the name of ``Frobenius algebras'').

After reviewing some basic vocabulary from category theory in Chapter~\ref{ch:cats}, we introduce oriented bordisms and their ambient category $ \mathbf{Bord}(n) $ in Chapter~\ref{ch:bord} along with the \emph{linear representations} of this category, i.e., the symmetric monoidal functors from $ \mathbf{Bord}(n) $ to the (symmetric monoidal) category of (finite-dimensional) vector spaces over an arbitrary field. Topological field theories arose in the eighties as an attempt to formalize some constructions appearing in quantum field theory. A TQFT assigns to each $(n-1)$-dimensional ``geometric object'' a vector space of ``states'', and to each ``worldvolume'' traced by a geometric object an ``evolution operator'' relating its initial and final states. The ``geometric objects'' we have in mind are $(n-1)$-dimensional manifolds and the ``worldvolumes'' they trace occur here as $n$-dimensional bordisms between them. While this intuitive picture alone legitimates a genuine interest in studying monoidal functors from categories of bordisms to categories of vector spaces, we will see that the rich geometric structure of the bordism category endows the category of its linear representation (i.e., a category of TQFTs) with nice algebraic properties. In Chapter~\ref{ch:bord2} we introduce the notion of generators and relations for a monoidal category, and, relying on the classification theorem for $2$-dimensional surfaces, we make the geometric structure alluded above explicit, presenting the $2$-dimensional bordism category concretely in terms of generators and relations. From the appropriate perspective, the relations presenting this bordism category look exactly like the axioms that define a Frobenius algebra. Chapter~\ref{ch:frobenius} develops a powerful graphical calculus for dealing with Frobenius algebras (and more generally, with algebraic structures defined within monoidal categories in general). We employ this calculus to analyze the multiple equivalent definitions of a Frobenius algebra, which can be described in so many words just as a monoid object in the category of vector spaces which is also a comonoid object in a compatible way. With these premises in place we finally reach the main theorem of this thesis in Chapter~\ref{ch:grand-finale}, which basically states that defining a $2$-dimensional TQFT is the same as choosing a commutative Frobenius algebra. We also see how this equivalence result is just an instance of a broader theorem: \bord{2} is the free symmetric monodial category over an internal Frobenius object.

The idea of writing a thesis on this topic arose from a wish to approach monoidal categories and their internal objects through a concrete example. The narrow time constraints determined the scope of this exposition, which offers nothing more than a quick informal route towards the equivalence theorem between $2$-dimensional TQFTs and Frobenius algebras. The serious reader is encouraged to turn (for instance) to~\cite{TuraevVirelizieMonoidal} (for the general theory), \cite{moore2025tasilecturestopologicalfield} (for a readable introduction to the physical side of the story), \cite{freed2013bordism} (for bordism theory), \cite{Atiyah1988} (for an original source) and of course to~\cite{kock2003frobenius}. The majority of this work clearly relies on \cite{kock2003frobenius} and can largely be seen as a retelling of its tale of \guillemotleft the commutative Frobenius and the princess $\mathbf{Bord}(2)$\guillemotright, even if by a less skillful bard.

\begin{flushright}
  Padova, December 2025
\end{flushright}

% defining free monoidal categories over some particular objects.

% our final step is 


% the final step needed to define an explicit link between bordisms and Frobenius algebras are representations of said bordisms. Indeed in Chapter~\ref{ch:grand-finale} we finally state that defining a 2-dimensional TQFT is the same as choosing a commutative Frobenius algebra (and viceversa).
% Those "Frobenius algebra-looking" relations come to life as Frobenius algebras when evaluated on the vector spaces. 

% prendono vita come algebre di frobenius quando le valutiamo su 

% This last notion

% This last notion is given for a reason: it is equivalent to the one of TQFTs.

 

% Things become even more concrete and structured in Chapter~\ref{ch:bord2} where we consider the particular case of dimension 2. Indeed, having an already established classification of topological surfaces allows us to give a presentation of the category \bord{2} in terms of generators and relations and ask ourselves what a representation of such category looks like. At the end we will discover this defines exactly a structure of Frobenius algebra. In Chapter~\ref{ch:frobenius} we'll have a close encounter with these algebras, through the help of a graphical language. 

% In laywoman's terms: the classification of $ (n - 1) $-dimensional manifolds up to isomorphism is a difficult task. Why not to relax the comparison relation to one which is easier to deal with (and at the same time not vacuous)?

% When trying to partition $(n-1)$-dimensional manifold into equivalence classes, instead of asking for two manifolds to be diffeomorphic, we ask if they together bound some $n$-dimensional manifold. This relation is broader than the one of being diffeomorphic (diffeomorphic manifolds are bordant), but way more practical. In this way we can forget the tedious details, while still remaining (deeply?) connected to topology and physics (and algebra). 


%%% Local Variables:
%%% mode: LaTeX
%%% TeX-master: "../main"
%%% End:

%% ASTRATTO
%% This thesis explores the equivalence between two dimensional topological quantum field theories and Frobenius algebras. We first provide a complete presentation by generators and relations of the $2$-dimensional oriented bordism category as a symmetric monoidal category. This presentation is then used to classify the $2$-dimensional topological field theories with values in an arbitrary target symmetric monoidal category.
 


% Capitolo 1
\fouche{bla}
\chapter{Categorical preliminaries}
\label{chap:cats}

We begin by recalling some useful notions from category theory, which we will encounter throughout this thesis, assuming the reader is already familiar with its fundamentals.

\section{Monoidal Categories}
\label{sec:moncat}
\begin{tcbdfn}[Monoidal categories] \label{dfn:monoidal_category}
 A (weak) \emph{monoidal category}\index{monoidal category} is a sextuple $(\cat{C}, \sq, \eta, \alpha, l, r)$ consisting of:
 \begin{itemize}
  \item a category $\cat{C}$,
  \item a bifuctor $\sq \colon \cat{C}\times\cat{C} \to \cat{C}$,
  \item a functor $\eta \colon \mathbf{1} \to \cat{C}$ identifying an object $\eta(1) = I \in \obj{\cat{C}}$\footnotemark,
  \item a natural isomorphism called \emph{associator} $\alpha$,
        \[
         \begin{tikzcd}
          & \cat{C} \times \cat{C} \times \cat{C}  \arrow[ld, "\sq \times \id_\cat{C}",swap] \arrow[rd,"\id_\cat{C} \times \sq"] & \\
          \cat{C} \times \cat{C} \arrow[rd, "\sq", swap] \arrow[rr, Rightarrow, "\alpha", shorten <=8ex, shorten >=8ex] & & \cat{C} \times \cat{C} \arrow[ld, "\sq"] \\
          & \cat{C} &
         \end{tikzcd}
        \]
        so that $\alpha_{A,B,C} \colon (A \sq B) \sq C \overset{\cong}\to A \sq (B \sq C)$ for every $A, B, C \in \obj{\cat{C}}$
  \item and two natural isomorphisms called \emph{left} and \emph{right unitors} $l$ and $r$
        \[
         \begin{tikzcd}
          & \cat{C} \times \cat{C} \arrow[rd, "\sq"] & & \cat{C}\times \cat{C} \arrow[ld, "\sq", swap]& \\
          \mathbf{1} \times \cat{C} \arrow[ru, "\eta \times \id_\cat{C}"] \arrow[rr, "\pi"{name=foot1}, swap]& & \cat{C} & & \cat{C} \times \mathbf{1} \arrow[lu, "\id_\cat{C} \times \eta", swap] \arrow[ll,"\pi"{name=foot2}]
          \arrow[Rightarrow, from=1-2, to=foot1, "l", shorten <=1ex, shorten >=1ex]  \arrow[Rightarrow, from=1-4, to=foot2, "r",shorten <=1ex,  shorten >=1ex]
         \end{tikzcd}
        \]
        so that $l_A \colon I \sq A  \overset{\cong}\to A$ and $r_A \colon A \sq I  \overset{\cong}\to A$ for every $A \in \obj{\cat{C}}$.
 \end{itemize}
 These maps must satisfy the so-called coherence requirements, depicted in \ref{fig:coherence_pentagon} and \ref{fig:coherence_unit}, for every $A,B,C,D \in \obj{\cat{C}}$.
\end{tcbdfn}
\footnotetext{We will hence equivalentely refer to a monoidal category as $(\cat{C}, \sq, I, \alpha, l, r)$}
\begin{figure}
 \centering
 \begin{tikzcd}[column sep=-2.75em, row sep=5em]
  &&(A \sq B) \sq (C \sq D) \arrow[rrd, "\alpha_{A, B, C \sq D}"] && \\
  ((A \sq B) \sq C) \sq D \arrow[rru, "\alpha_{A \sq B, C, D}"] \arrow[rdd, "\alpha_{A,B,C} \sq \id_D",swap] &&&& A \sq (B \sq (C \sq D)) \\
  &&&& \\
  & (A \sq (B \sq C)) \sq D \arrow[rr, "\alpha_{A, B \sq C, D}"] && A \sq((B \sq C) \sq D) \arrow[ruu, "\id_A \sq \alpha_{B, C, D}",swap] &
 \end{tikzcd}
 \caption{Associativity coherence}
 \label{fig:coherence_pentagon}
\end{figure}
\begin{figure}
 \centering
 \begin{tikzcd}
  (A \sq I) \sq B \arrow[rr, "\alpha_{A,I,B}"] \arrow[dr, "r_A \sq \id_B",swap] && A \sq(I \sq B) \arrow[dl, "\id_A \sq l_B"] \\
  & A \sq B &
 \end{tikzcd}
 \caption{Unit coherence}
 \label{fig:coherence_unit}
\end{figure}

It is important to notice that naturality of such transformations is not for free, as some proof depend on naturality to work.
\begin{tcbdfn}[Strict monoidal categories]
 A monoidal category is \emph{strict}\index{monoidal category!strict monoidal category} if all the natural transformations $\alpha$, $l$, $r$ are identities.
\end{tcbdfn}

We can say that a strict monoidal category is a category $\cat{C}$ equipped with an associative bifunctor $\sq \colon \cat{C} \times \cat{C} \to \cat{C}$ and an object $I \in \obj{\cat{C}}$ which is a left and a right unit for $\sq$. In fancy words, a strict monoidal category is a (strict) monoid \emph{internal} to the category of categories and functors.
\par
From now on, when considering a weak monoidal category $(\cat{C}, \sq, I, \alpha^\cat{C}, l^\cat{C}, r^\cat{C})$, we will refer to it with using the notation $(\cat{C}, \sq, I)$. This is an abuse of notation that allows for brevity.
\begin{tcbprp}\label{prp:cartesian-monoidal}
 Let $\cat{C}$ be a category that admits products. Then $(\cat{C}, \times, 1)$, where $\times$ denotes the product and $1$ the terminal object, is a (weak) monoidal category.
\end{tcbprp}
\begin{proof}
 Consider $A, B , C \in \obj{\cat{C}}$. By the universal property of the product, we can define an isomorphism $\alpha_{A, B, C}$ between $(A \times B) \times C$ and $A \times (B \times C)$, with which we can define the \emph{associator} $\alpha$. Then, since for any $A \in \obj{\cat{C}}$ there exists a unique arrow to the terminal object $1$, we have that
 \[
  \begin{tikzcd}
   1 & A \arrow[l, "\text{!}",swap] \arrow[r, "\id_A"] & A
  \end{tikzcd}
 \]
 defines a product for $1$ and $A$. The uniqueness of the product provides an isomorphism $1 \times A \iso A$ and, similarly, $A \times 1 \iso A$. Through such maps, we define \emph{left} and \emph{right unitors}. Using such isomorphisms one can also check that coherence holds.
%  \fouche{Lo hai fatto? Cioè, sei capace di spiegare a parole perché il pentagono commuta? Come sono definiti associatore e unitore, è veramente naturale?}
\end{proof}
We call this structure a \emph{cartesian} monoidal category. The dual statement also holds.
\begin{tcbprp}\label{prp:cocartesian-monoidal}
 Let $\cat{C}$ be a category that admits coproducts. Then $(\cat{C}, +, 0)$, where $+$ denotes the coproduct and $0$ the initial object, is a (weak) monoidal category.
\end{tcbprp}
We call this structure a \emph{cocartesian} monoidal category.
\begin{tcbex}
 Examples of monoidal categories, which we will encounter throughout this thesis, are:
 \begin{itemize}
  \item $(M, \mu, 1_M)$, the discrete monoidal category on a monoid $M$
  \item $(\mathbf{Set}, {\times}, \{\ast\})$, the cartesian monoidal category of sets and functions between them
  \item $(\mathbf{Set}, \disj, \varnothing)$, the cocartesian monoidal category of sets and functions between them
  \item $(\mathbf{Cat}, {\times}, \mathbf{1})$, the cartesian monoidal category of categories and functors between them
  \item $(\Delta, {+}, [0])$, the strict monoidal simplicial category. A more in-depth description of it can be found in \ref{dfn:delta-cat}.
  \item $(\mathbf{Vect}_\Bbbk, {\tensor}, \Bbbk)$, the monoidal category of vector spaces over the field $\Bbbk$ and linear maps between them, with the monoidal structure given by the tensor product
 \end{itemize}
\end{tcbex}
We'd now like to define morphisms between monoidal categories as functors that preserve the monoidal structure. However, we need to be precise about the degree to which such a structure needs to be preserved.
As for monoidal categories, the weakest definition will also be the most intricate.

\begin{tcbdfn}[Monoidal functors]\label{dfn:monoidal-functor}
 Let $(\cat{C}, \sq, I)$ and $(\cat{D}, \varsq , J)$ be two (weak) monoidal categories. A monoidal functor is a triple $(F, \varphi, \varepsilon)$ consisting of
 \begin{itemize}
  \item a functor \[ F \colon \cat{C} \to \cat{D} \]
  \item a natural transformation
        %       \[
        %        \begin{tikzcd}
        %        & \cat{C} \times \cat{C} \arrow[ld, "F \times F",swap] \arrow[rd,"\sq"] & \\
        %        \cat{D} \times \cat{D} \arrow[rd, "\varsq", swap] \arrow[rr, Rightarrow, "\varphi", shorten <=6ex, shorten >=6ex] & & \cat{C} \arrow[ld, "F"] \\
        %        & \cat{D} &
        %        \end{tikzcd}
        %        \]
        %        which gives us 
        \[ \varphi_{A,B} \colon F(A) \varsq F(B) \to F(A \sq B)\] for every $A,B \in \obj{\cat{C}}$
  \item a morphism \[ \varepsilon \colon J \to F(I) \]
 \end{itemize}
 making the following diagrams commute
 \begin{enumerate}
  \item \emph{(Associativity)}
        \[
         \begin{tikzcd}[column sep=6em]
          (F(A) \varsq F(B)) \varsq F(C) \arrow[r, "\alpha^\cat{D}_{F(A), F(B),F(C)}"]  \arrow[d, "\varphi_{A,B} \varsq \id_{F(C)}"]
          & F(A) \varsq (F(B) \varsq F(C)) \arrow[d, "\id_{F(A)} \varsq \varphi_{B,C}"] \\
          F(A \sq B) \varsq F(C) \arrow[d] \arrow[d, "\varphi_{A \sq B, C}"]
          & F(A) \varsq F(B \sq C) \arrow[d, "\varphi_{A, B \sq C}"]
          \\
          F((A \sq B) \sq C) \arrow[r, "F(\alpha^\cat{C}_{A,B,C})"]
          & F(A \sq (B \sq C))
         \end{tikzcd}
        \]
  \item \emph{(Unitality)}
        \[
         \begin{tikzcd}
          F(A) \varsq J \arrow[r, "r^\cat{D}_{F(A)}"] \arrow[d, "\id_{F(A)} \varsq \varepsilon",swap] & F(A) \\
          F(A) \varsq F(I) \arrow[r, "\varphi_{A,I}"] & F(A \sq I) \arrow[u, "F \circ r^\cat{C}_A",swap]
         \end{tikzcd}
         \qquad
         \begin{tikzcd}
          J \varsq F(A) \arrow[r, "l^\cat{D}_{F(A)}"] \arrow[d, "\varepsilon\varsq \id_{F(A)}",swap] & F(A) \\
          F(I) \varsq F(A) \arrow[r, "\varphi_{I,A}"] & F(I \sq A) \arrow[u, "F \circ l^\cat{C}_A",swap]
         \end{tikzcd}
        \]
 \end{enumerate}
 for every $A,B,C \in \obj{\cat{C}}$
\end{tcbdfn}

\begin{remark}
 This notion can be found in the literature as \emph{lax monoidal functor}.
 \fouche{add cit?}
 We can also define an \emph{oplax monoidal functor} between two categories $\cat{C}$ and $\cat{D}$ as a lax monoidal functor $\cat{C}^{op} \to \cat{D}^{op}$.
 In this case, we will have natural transformations $\varphi_{A, B} \colon F(A \sq B) \to F(A) \varsq F(B)$ and $\varepsilon \colon F(I) \to J$. The idea behind these two notions is that a \emph{lax} monoidal functor is the one preserving the structure of an internal monoid, while an \emph{oplax} monoidal functor preserves the internal comonoid structure. The definition of an internal monoid (resp. comonoid) will be properly given in \ref{dfn:internal-monoid} (resp. \ref{dfn:internal-comonoid}).
\end{remark}

\begin{tcbdfn}[Strong monoidal functors]
 Let $(\cat{C}, \sq, I)$ and $(\cat{D}, \varsq , J)$ be two (weak) monoidal categories. A monoidal functor $(F, \varphi, \varepsilon)$ from $\cat{C}$ to $\cat{D}$ is \emph{strong monoidal}\index{monoidal category!strong monoidal functor} if $\varphi$ and $\varepsilon$ are isomorphisms.
\end{tcbdfn}

We hence have $F(A) \varsq F(B) \iso F(A \sq B)$ and $J \iso F(I)$.

\begin{tcbdfn}[Strict monoidal functors]
 Let $(\cat{C}, \sq, I)$ and $(\cat{D}, \varsq , J)$ be two (weak) monoidal categories. A monoidal functor $(F, \varphi, \varepsilon)$ from $\cat{C}$ to $\cat{D}$ is \emph{strict monoidal}\index{monoidal category!strict monoidal functor} if $\varphi$ and $\varepsilon$ are identities.
\end{tcbdfn}

The conditions for a strict monoidal functor are asking that
\[
 F (A \sq B) = F(A) \varsq F(B) \text{,} \qquad F(I) = J
\]
\[
 F(f \sq g) = F(f) \varsq F(g)
\]
\[
 F(\alpha^\cat{C}_{A,B,C}) = \alpha^\cat{D}_{F(A), F(B), F(C)} \qquad F(l^\cat{C}_A) = l^\cat{D}_{F(A)} \qquad F(r^\cat{C}_A) = r^\cat{D}_{F(A)}
\]
for every $A, B, C \in \obj{\cat{C}}$ and $f, g \in \arr{\cat{C}}$.

\begin{remark}
 Let $(\cat{C}, \times_\cat{C}, 1_\cat{C})$, $(\cat{D}, \times_\cat{D}, 1_\cat{D})$ be two cartesian monoidal categories and $F \colon \cat{C} \to \cat{D}$ a functor between them. By the universal property of the terminal object, there exists a unique arrow $\varepsilon \colon F(1_\cat{C}) \to 1_\cat{D}$, and by the universal property of the product, we have
 \[
  \begin{tikzcd}[column sep=large]
   F(A) & F(A) \times_\cat{D} F(B) \arrow[l, "\pi_{F(A)}", swap] \arrow[r, "\pi_{F(B)}"] & F(B) \\
   & F(A \times_\cat{C} B) \arrow[ul, "F(\pi_A)"] \arrow[ur, "F(\pi_B)", swap] \arrow[u, dashed, "\exists !"] &
  \end{tikzcd}
 \]
 Again, by relying on universal properties, we are able to prove that associativity and unitality hold. We hence have that $F \colon \cat{C} \to \cat{D}$ is \emph{oplax} monoidal without further requirements. Similarly, every functor $F \colon \cat{C}^{op} \to \cat{D}^{op}$ is automatically \emph{lax} monoidal.

 Conversely, let $(\cat{C}, +_\cat{C}, 0_\cat{C})$, $(\cat{D}, +_\cat{D}, 0_\cat{D})$ be two cocartesian monoidal categories and $F \colon \cat{C} \to \cat{D}$ a functor between them. Then $F$ is a \emph{lax} monoidal functor. Similarly, every functor $F \colon \cat{C}^{op} \to \cat{D}^{op}$ is automatically \emph{oplax} monoidal.
\end{remark}


The composition of two (lax, oplax, strong, strict) monoidal functors is again a (lax, oplax, strong, strict) monoidal functor, and identity functors are monoidal. We hence have a category $\moncat_{\textup{lax}}$ where objects are (weak) monoidal categories and arrows are lax monoidal functors. Similarly, $\moncat_{\textup{strong}}$ defines the category whose objects are (weak) monoidal categories and arrows are strong monoidal functors. There is also a full subcategory where objects are strict monoidal categories and arrows are strong monoidal functors.

\begin{remark}\label{rmk:strictification}
 As the reader can imagine, working with weak monoidal categories can become a really tedious process. The following result mitigates this difficulty, justifying working with monoidal categories \emph{as if they were strict}, forgetting about associators and unitors without losing generality. Its proof and a more in-depth explanation can be found in \cite{maclane} and \cite{coherence}.
\end{remark}
\begin{tcbthm}[Strictification]\label{thm:strictification}
 \emph{(See \cite{maclane}, Chapter XI, Section 3)} Every monoidal category is categorically equivalent, via strong monoidal functors, to a strict monoidal category.
\end{tcbthm}
Essentially, what follows is that from now on we can assume, without loss of generality, that monoidal categories \emph{are} strict. This is possible because the equivalence provided by the above theorem ensures that all coherence diagrams commute up to a unique canonical isomorphism, as long as we do not require functors between such categories to be strict. To do so, from now on, when considering the functor category $\mathbf{MonCat}$, we are referring to $\mathbf{MonCat}_{\textup{strong}}$.
\begin{remark}
 The strictification of the monoidal category of finite dimensional vectors spaces over a field $k$ is the category (cf. \cite[2.§2]{Walters1992}) $\mathbb{N}[k]$ whose objects are iterated powers $k^n$ of $k$ with itself; the monoidal structure is given by product of natural numbers, i.e. as the strict equality 
 \[k^n\oplus k^m = k^{nm}\]
 Note that $\mathbb{N}[k]$ is a PROP in the sense of \cite[Ch. V, §24]{MacLane1965}; the sum of natural numbers gives $\mathbb{N}[k]$ another monoidal structure $\oplus$, in that 
 \[k^n \oplus k^m = k^{n+m}\]
 and moreover $\mathbb{N}[k]$ is a categorified version of a \emph{semiring}, in that `product distributes over sum', 
 \[k^{n(p+q)} = k^{np}\oplus k^{nq}.\]
\end{remark}
% \todo[inline]{idee: inserire esempio costruzione tensore n esimo. inserire esempio strictificazione di Vect}

% \begin{tcbex}
%     Consider the (weak) monoidal category $(\mathbf{Vect}_\Bbbk, \tensor, \Bbbk)$. By taking 
% \end{tcbex}

\begin{tcbdfn}[Monoidal natural transformations]
 Let $(\cat{C}, \sq, I)$ and $(\cat{D}, \varsq, J)$ be two monoidal categories and let $(F, \varphi, \varepsilon)$ and $(G, \phi, e)$ be two monoidal functors from $\cat{C}$ to $\cat{D}$. A \emph{monoidal natural transformation}\index{monoidal category!monoidal natural transformation} is a natural transformation $u \colon F \to G$ such that the following diagrams commute
 \[
  \begin{tikzcd}
   F(A) \varsq F(B) \arrow[r, "u_A \varsq u_B"] \arrow[d, "\varphi_{A,B}"] & G(A) \varsq G(B) \arrow[d, "\phi_{A,B}"] \\
   F(A \sq B) \arrow[r, "u_{A \sq B}"] & G(A \sq B)
  \end{tikzcd}
  \qquad
  \begin{tikzcd}
   J \arrow[r, "\varepsilon"] \arrow[dr, "e", swap] & F(I) \arrow[d, "u_I"] \\
   & G(I)
  \end{tikzcd}
 \]
 for every $A,B \in \obj{\cat{C}}$.
\end{tcbdfn}

\begin{remark}
 Let $(\cat{C}, \times_\cat{C}, 1_\cat{C})$, $(\cat{D}, \times_\cat{D}, 1_\cat{D})$ be two cartesian monoidal categories and let $F \colon \cat{C} \to \cat{D}$, $G \colon \cat{C} \to \cat{D}$ be two functors between them. Consider any natural transformation $u \colon F \Rightarrow G$ and the diagrams
 \[
  \begin{tikzcd}
   F(A) \times_\cat{D} F(B) \arrow[r, "u_A \times_\cat{D} u_B"] & G(A) \times_\cat{D} G(B) \\
   F(A \times_\cat{C} B) \arrow[u, "\varphi_{A,B}"] \arrow[r, "u_{A \times_\cat{C} B}"] & G(A \times_\cat{C} B) \arrow[u, "\phi_{A,B}"]
  \end{tikzcd}
  \qquad
  \begin{tikzcd}
   1_\cat{C} & F(1_\cat{C}) \arrow[l, "\varepsilon", swap] \arrow[d, "u_I"] \\
   & G(1_\cat{C}) \arrow[ul, "e"]
  \end{tikzcd}
 \]
 By naturality of $u$, one checks the two diagrams commute without asking further conditions to hold. We hence have that every natural transformation between two functors defined on cartesian monoidal categories is monoidal. The same holds when the two categories are cocartesian monoidal.
\end{remark}

\section{Adding a twist}
\label{sec:twist}

\begin{tcbdfn}[Braided monoidal categories]
 A \emph{braided monoidal category}\index{braided monoidal category} is a monoidal category $(\cat{C}, \sq, I)$ equipped with a natural isomorphisms
 \[
  \beta_{A,B} \colon A \sq B \to B \sq A
 \]
 for every $A,B \in \obj{\cat{C}}$, called the \emph{braiding}, such that the following diagrams commute
 \[
  \begin{tikzcd}
   & A \sq (B \sq C) \arrow[r, "\beta_{A, B \sq C}"] & (B \sq C) \sq A \arrow[dr, "\alpha_{B, C, A}"] & \\
   (A \sq B) \sq C \arrow[ur, "\alpha_{A, B, C}"] \arrow[dr, "\beta_{A, B} \sq \id_{C}", swap] &&& B \sq (C \sq A) \\
   & (B \sq A) \sq C \arrow[r, "\alpha_{B, A, C}"] & B \sq (A \sq C) \arrow[ur, "\id_B \sq \beta_{A, C}", swap]
  \end{tikzcd}
 \]
 \[
  \begin{tikzcd}
   & (A \sq B) \sq C \arrow[r, "\beta_{A \sq B, C}"] & C \sq (A \sq B) \arrow[dr, "\alpha^{-1}_{C,A,B}"] & \\
   A \sq (B \sq C) \arrow[ur, "\alpha^{-1}_{A,B,C}"] \arrow[dr, "\id_A \sq \beta_{B, C}", swap] &&& (C \sq A) \sq B \\
   & A \sq (C \sq B) \arrow[r, "\alpha^{-1}_{A, C, B}"] & (A \sq C) \sq B \arrow[ur, "\beta_{A, C} \sq \id_A", swap]
  \end{tikzcd}
 \]
\end{tcbdfn}

The definition implies compatibility with the unital structure. For a proof of the following result, we refer to \cite{kock2003frobenius}.
\begin{tcbprp}
 Let $(\cat{C}, \sq, I, \beta)$ be a braided monoidal category. Then the following diagrams commute for every $A \in \obj{\cat{C}}$.
 \[
  \begin{tikzcd}
   A \sq I \arrow[rr, "\beta_{A,I}"] \arrow[dr, "r_A", swap] && I \sq A \arrow[dl, "l_A"] \\
   & A &
  \end{tikzcd}
  \qquad
  \begin{tikzcd}
   I \sq A \arrow[rr, "\beta_{I,A}"] \arrow[dr, "l_A", swap] && A \sq I \arrow[dl, "r_A"] \\
   & A &
  \end{tikzcd}
 \]
\end{tcbprp}

Again, by \ref{thm:strictification}, when considering braided monoidal categories, we can restrict to the following definition.

\begin{tcbdfn}[(Semi)strict braided monoidal categories]
 A semistrict \emph{braided monoidal category}\index{braided monoidal category!semistrict braided monoidal category} is a strict monoidal category $(\cat{C}, \sq, I)$ equipped with a natural isomorphism
 \[
  \beta_{A,B} \colon A \sq B \to B \sq A
 \]
 for every $A, B \in \obj{\cat{C}}$, called the \emph{braiding}, such that the following diagrams commute
 \[
  \begin{tikzcd}[column sep=-.5em]
   A \sq B \sq C \arrow[rr, "\beta_{A, B \sq C}"] \arrow[dr, "\beta_{A, B} \sq \id_C", swap] & & B \sq C \sq A \\
   & B \sq A \sq C \arrow[ur, "\id_B \sq \beta_{A, C}", swap] &
  \end{tikzcd}
  \qquad
  \begin{tikzcd}[column sep=-.5em]
   A \sq B \sq C \arrow[rr, "\beta_{A \sq B, C}"] \arrow[dr, "\id_A \sq \beta_{B, C}", swap] && C \sq A \sq B \\
   & A \sq C \sq B \arrow[ur, "\beta_{A, C} \sq \id_A", swap] &
  \end{tikzcd}
 \]
\end{tcbdfn}

\begin{remark}
 It would be misleading to name these categories ``strict braided monoidal''. While the underlying monoidal structure is strict, we are not asking for the braiding itself to be strict.
\end{remark}

\begin{tcbdfn}[Symmetric monoidal categories]
 A \emph{symmetric monoidal category}\index{symmetric monoidal category} is a braided monoidal category $(\cat{C}, \sq, I, \beta)$ where the braiding $\beta$ is symmetric, that is
 \[\beta_{B,A} \circ \beta_{A,B} = \id_{A \sq B}\]
 for every $A, B \in \obj{\cat{C}}$.
\end{tcbdfn}

Let us show some examples of symmetric monoidal categories.

\begin{tcbex}
 The monoidal category $(\mathbf{Vect}_\Bbbk, \tensor, \Bbbk)$ is equipped with a canonical symmetry defined by
 \begin{align*}
  \sigma \colon V \tensor W & \to W \tensor V \\ v \tensor w & \mapsto w \tensor v
 \end{align*}
\end{tcbex}

\begin{tcbex}[Free symmetric monoidal category on a single object]
 We define the \emph{symmetric monoidal category} $\mathbf{Sym}$ by considering
 \begin{itemize}
  \item the sets of natural numbers $[n] = {0 ,\dots, n-1}$ as objects. We also define $[0] = \varnothing$.
  \item the elements of the symmetric group $\text{Sym}(n)$ as morphism $\Hom([n], [n])$. When $[n] \neq [m]$, we have $\Hom([m],[n]) = \varnothing$.
 \end{itemize}
 The monoidal operation $+$ is defined by juxtaposition\footnotemark, with the empty set $\varnothing = [0]$ as the unit object. The symmetric structure is defined by interchanging factors (we define the \emph{twist} from $[m]+[n]$ to $[n]+[m]$ by the association $\{0_m, \dots, (m-1)_m, 0_n, \dots,  (n-1)_n\} \mapsto \{0_n, \dots, (n-1)_n, 0_m, \dots, (m-1)_m\}$).
\end{tcbex}
\footnotetext{We can see an explicit definition of such operation in \ref{dfn:ordinal-sum}, for the case of $\Delta$}

\begin{tcbex}[Free braided monoidal category on a single object]
 Similarly to the previous case, we define the \emph{braided monoidal category} $(\mathbf{Braid}, +, [0])$ by taking the Artin braid groups $\text{Braid}(n)$ (see \cite{Kassel2008})a s morphisms.
\end{tcbex}

It is important to understand that being a symmetric monoidal category involves defining a structure, not merely possessing a property. This distinction is clearly shown by the following example.
\begin{tcbex}[A category with more than one symmetric structure]
 Let $\mathbf{grVect}_\Bbbk$ be the category of \emph{graded vector spaces} (i.e. direct sums of vector spaces $V = \bigoplus_{n \in \mathbb{Z}} V_n$ together with linear maps $f = \bigoplus_{n \in \mathbb{Z}}f_n$ respecting the grading)\footnotemark. The monoidal structure defined for $\mathbf{Vect}$ restricts to the graded setting and defines the monoidal category $(\mathbf{grVect}_\Bbbk, \tensor, \Bbbk)$.

 In this setting, we have a canonical symmetry $v \tensor w \mapsto w \tensor v$, but we can also define a different isomorphism, known as the \emph{Koszul's sign change}
 \begin{align*}
  \kappa \colon V \tensor W & \to W \tensor V                                           \\
  v \tensor w               & \mapsto (-1)^{\mathrm{deg}(v)\mathrm{deg}(w)} w \tensor v
 \end{align*}
 Both $(\mathbf{grVect}_\Bbbk, \tensor, \Bbbk, \sigma)$ and $(\mathbf{grVect}_\Bbbk, \tensor, \Bbbk, \kappa)$ are symmetric monoidal categories, yet they are distinct and carry different internal structures.
\end{tcbex}
\footnotetext{What we are describing are actually $\mathbb{Z}$-graded vector spaces. In general, we can define $G$-graded vector spaces as $V = \bigoplus_{g \in G} V_g$ for any group $G$.}

\begin{tcbdfn}[Braided monoidal functors]
 Let $(\cat{C}, \sq, I, \beta)$ and $(\cat{D}, \varsq, J, \gamma)$ be two braided monoidal categories. A monoidal functor $(F, \varphi, \varepsilon)$ from $\cat{C}$ to $\cat{D}$ is a \emph{braided monoidal functor}\index{braided monoidal category!braided monoidal functor} if the diagram
 \[
  \begin{tikzcd}[column sep=huge]
   F(A) \varsq F(B) \arrow[r, "\gamma_{F(A), F(B)}"] \arrow[d, "\varphi_{A,B}"] & F(B) \varsq F(A) \arrow[d, "\varphi_{B,A}"] \\
   F(A \sq B) \arrow[r, "F \circ \beta_{A,B}"] & F(B \sq A)
  \end{tikzcd}
 \]
 commutes for every $A, B \in \obj{\cat{C}}$.
\end{tcbdfn}

When the braiding $\beta$ is symmetric, we say the functor is \emph{symmetric monoidal}.

As we noticed for monoidal categories, we have that the composition of braided (resp. symmetric) monoidal functors is a braided (resp. symmetric) monoidal functor, and the identity monoidal functor is braided (resp., symmetric).
We can hence define the categories $\brmoncat$ and $\symmoncat$. % where objects are respectively braided and symmetric monoidal categories, while arrows are braided and symmetric monoidal functors.
Again, similarly to the plain monoidal case, given two braided (resp. symmetric) monoidal categories $(\cat{C}, \sq, I, \beta)$, $(\cat{D}, \varsq, J, \gamma)$ we can consider the category $\brmoncat(\cat{C},\cat{D})$\index{braided monoidal category!braided monoidal functor category} (resp. $\symmoncat(\cat{C}, \cat{D})$\index{symmetric monoidal category!symmetric monoidal functor category}) where objects are braided (resp. symmetric) monoidal functors form $\cat{C}$ to $\cat{D}$ and arrows are monoidal natural transformations between such functors.
%\section{Monoidal functor category}
%It is easy to see that the definitions just given let us define the categories

\section{Rigid categories}\label{sec:rigid}

We include a brief treatment of the concept of duals in a monoidal category, a notion that will recur again throughout this thesis. We refer to \cite{Selinger2010}.

\begin{tcbdfn}[Exact pairing, left and right duals]
      Let $(\cat{C}, \sq, I)$ be a (without loss of generalities, strict) monoidal category. An \emph{exact pairing}\index{exact pairing} between two objects $A$ and $B$ is given by a pair of morphisms $\eta \colon I \to B \tensor A$ and $\varepsilon \colon A \tensor B \to I$, such that the following triangles commutes
      \[
      \begin{tikzcd}
            A \arrow[r, "\id_A \tensor \eta"] \arrow[dr, "\id_A", swap] & A \tensor B \tensor A \arrow[d, "\varepsilon \tensor \id_A"]\\
            & A
      \end{tikzcd}
      \hspace{5em}
      \begin{tikzcd}
            B \arrow[r, "\eta \tensor \id_B"] \arrow[dr, "\id_B", swap] & B \tensor A \tensor B \arrow[d, "\id_B \tensor \varepsilon"] \\
            & B
      \end{tikzcd}
      \]
      In such exact pairing, the object $B$ is called \emph{right dual} of $A$ and $A$ is called \emph{left dual} of $B$.
\end{tcbdfn}

When the monoidal category is symmetric, left and right duals coincide; we refer to either as the \emph{dual}\index{dualizability!dualizable objects}.We call every object that admits categorical duals \emph{fully dualizable}.

\begin{tcbdfn}[Rigid categories]
      A monoidal category $(\cat{C}, \sq, I)$ is called \emph{rigid}\index{rigid category} if every of its object has both left and right duals.
\end{tcbdfn}



% Capitolo 2
\chapter{Bordisms and TQFTs}
\label{chap:bord}

We denote with $\mathbb{R}^n$ the $n$-dimensional Euclidean space composed of all ordered $n$-tuples $(x^1,.....,x^n)$ of real numbers. We denote with $\mathbb{H}^n$ the set $\{ x \in \mathbb{R}^n \colon x^n \geq 0 \}$ equipped with the topology induced by $\mathbb{R}^n$. Recall that an $n$-dimensional \emph{topological manifold} is a second-countable Hausdorff topological space locally homeomorphic to $\mathbb{R}^n$. An $n$-dimensional \emph{topological manifold with boundary} is a second-countable Hausdorff topological space locally homeomorphic to $\mathbb{H}^n$. When we say that we have a \emph{local chart} on a topological manifold $M$, we are referring to a pair $(U, u)$ where $U \subseteq X$ is an open subset of $X$ and $u \colon U \to \underline{U} \subseteq \mathbb{R}^n$ is an homeomorphism. Two charts $(U,u)$, $(V,v)$ on $M$ are \emph{($C^\infty$)-compatible} if $ U\cap V = \emptyset $ or if $ U\cap V\neq \emptyset $ and the transition maps $u \circ v^{-1}$ and $v \circ u^{-1}$ are infinitely differentiable. An \emph{atlas} $\mathcal{A}$ on $M$ is a collection $\{(U_i, u_i)\}_{i \in I}$ of compatible charts such that $\{U_i\}_{i \in I}$ is an open covering of $M$. Two atlases $\mathcal{A}$, $\mathcal{B}$ are \emph{compatible} if $\mathcal{A} \cup \mathcal{B}$ is an atlas. A \emph{smooth ($C^\infty$-)manifold} is a second-countable Hausdorff topological space $M$ equipped with a \emph{smooth structure}, i.e. an equivalence class of compatible atlases. In the same vein, we can define a \emph{smooth manifold with boundary}.  A point $x \in M$ is a \emph{boundary point} if it is mapped, through some local chart, into the boundary $\{ x \in \mathbb{R}^n \colon x^n = 0\}$ of $\mathbb{H}^n$.  The set of all boundary points of an $n$-dimensional manifold $M$ is an $(n-1)$-dimensional manifold we'll denote with $\partial M$, the \emph{boundary} of $M$. The boundary of a manifold can also be empty. In this way, every manifold can be seen as a manifold with boundary. In the following, a compact manifold with no boundary is a \emph{closed manifold}.


\section{Bordisms}

Having recalled the basic framework, we continue with some more precise notions from differential geometry. We'll assume all manifolds to be $C^\infty$.

\begin{tcbdfn}[Orientation of a vector space]
 Let $V$ be a finite-dimensional real vector space. We say that two ordered bases $\mathcal{B}_1$, $\mathcal{B}_2$ have the \emph{same orientation} (resp. \emph{opposite orientation}) if the linear transformation carrying one into the other has positive (resp. negative) determinant. An \emph{orientation} on $V$ is given by associating a sign (either $+$ or $-$) to each ordered basis, following the rule just stated.
\end{tcbdfn}

\begin{tcbdfn}[Linear maps and orientation]
 Let $V$, $W$ be two oriented, finite-dimensional real vector spaces. A linear map $f \colon V \to W$ is \emph{orientation preserving} if it sends positive bases into positive bases and \emph{orientation reversing} if positive bases are sent into negative ones.
\end{tcbdfn}

% \begin{tcbdfn}[Oriented manifolds]
% An \emph{orientation} of a (smooth) manifold $M$ is a choice of orientation of its tangent bundle $TM$.  In making such a choice we require that the differentials of the transition functions preserve orientations.  We say a smooth manifold $M$ is \emph{orientable} when it admits an orientation.
% \end{tcbdfn}
\begin{tcbdfn}[Oriented manifolds]
 An \emph{orientation} of a (smooth) manifold $M$ is a choice of orientation for each tangent space $T_xM$. In making such a choice, we require that the differentials of the transition functions preserve orientations. We say a smooth manifold $M$ is \emph{orientable} when it admits an orientation.

 A differentiable map $f \colon M \to M'$ between two oriented (smooth) manifolds is \emph{orientation preserving} if, for each $x \in M$, the differential $df_x \colon T_xM \to T_{f(x)}M'$ is orientation preserving.
\end{tcbdfn}

% \todo[inline, color=yellow]{{\bfseries Marco cattivo}: Nn è vero.}

Each orientable connected smooth manifold admits two possible orientations.  An orientable manifold with $k \geq 0$ connected components admits $2^k$ possible orientations. The empty manifold has exactly one orientation.

\begin{tcbdfn}[Orientation of a product]
 Let $M$, $N$ be two oriented manifolds, where at least one of them has no boundary. The product $M \times N$ acquires an orientation such that for each point $(x, y) \in M \times N$, if $\{v_1, \dots, v_m \}$ is a positive basis for $T_x(M)$ and $\{w_1, \dots, w_n\}$ is a positive basis for $T_y(N)$, then $\{v_1, \dots, v_m,w_1, \dots, w_n\}$ is a positive basis for $T_{(x,y)}(M \times N)$.
\end{tcbdfn}

\begin{example}
 Take a circle $\mathbb{S}^1$ with the usual counterclockwise orientation and the interval $I \defeq [0,1]$ with its standard orientation. The products $\mathbb{S}^1 \times I$ and $I \times \mathbb{S}^1$ then have opposite orientations. We hence need to be careful when choosing the order of such factors.

 \[
  \begin{tikzpicture}[every tqft/.style={transform shape}, rotate=90, scale=1.5]
   \pic[tqft,
    incoming boundary components=1,
    outgoing boundary components=1,
    draw,
    every outgoing lower boundary component/.style={draw},
    every incoming lower boundary component/.style={dashed,draw},
    name=a,
    anchor={(0,0)}
   ];
   \node at (1,-1) {$\mathbb{S}^1 \times I$};
   \node at (0.5,-1) {$\{s, e\}$ positive basis for $T_x(\mathbb{S}^1 \times I)$};
   \pic[tqft,
    incoming boundary components=1,
    outgoing boundary components=1,
    draw,
    every outgoing lower boundary component/.style={draw},
    every incoming lower boundary component/.style={dashed,draw},
    name=b,
    anchor={(0,-2.5)}
   ];
   \node at (1,-6) {$I \times \mathbb{S}^1$};
   \node at (0.5,-6) {$\{e, s\}$ positive basis for $T_x(I \times \mathbb{S}^1)$};
   \draw [->] (2,-1) -- node [above=.15em,align=right] {$e$} (2,-1.5);
   \draw [->] (2,-1) -- node [left=.1em,align=center] {$s$} (1.5,-1);
   \draw [->, ibmBlue] (1.75, -1) to[out=180,in=180] (2,-1.25);
   \draw [->] (2,-6) -- node [above=.15em,align=right] {$e$} (2,-6.5);
   \draw [->] (2,-6) -- node [left=.15em,align=center] {$s$} (1.5,-6);
   \draw [->, ibmBlue] (2,-6.25) to[out=180,in=-90] (1.75,-6);
  \end{tikzpicture}
 \]
\end{example}

\begin{tcbdfn}
 Let $M$ be a manifold with boundary and $p\in \partial M$. A vector $v \in T_p M \setminus T_p (\partial M)$ is \emph{inward-pointing} if for some $\varepsilon > 0$ there exists a smooth curve $\gamma \colon [0, \varepsilon) \to M$ such that $\gamma(0) = p$ and $\gamma'(0) = v$.
    A vector $v \in T_p M \setminus T_p (\partial M)$ is \emph{outward-pointing} if there exists such a curve whose domain is $(-\varepsilon, 0]$.
\end{tcbdfn}

By considering some smooth chart $(U \ni p, \varphi = (x^i))$, we then have that the inward-pointing vectors in $T_p M$ are exactly the ones with $x^n > 0$ and the outward-pointing ones are those for which $x^n < 0$. To avoid confusion, we remember that this correspondence depends on the space $\mathbb{H}^n$ we chose when defining manifolds with a boundary, in this case $\mathbb{H}^n = \{x^n \geq 0\}$.

% \todo[inline, color=yellow]{{\bfseries Marco cattivo}: Spiega che le nozioni di inward/outward pointing dipendono dallo spazio modello che hai scelto per le varietà con bordo. Nel tuo caso hai deciso che $ \mathbb H^n = \{x^n\geq 0\} $, quindi è giusto.}

% ref: dfn 15.24 Lee
\begin{tcbdfn}[The induced orientation on a boundary]
 Let $M$ be an oriented manifold with boundary. Its boundary $\partial M$ inherits a canonical orientation\footnotemark, defined as follows. Consider an outward pointing vector $n \in T_x M \setminus T_x (\partial M)$; for any basis $\{ t_1, \dots, t_n\}$ of $T_x (\partial M)$ we say it to be a $\emph{positive}$ (resp. \emph{negative}) if $\{n, t_1, \dots, t_n\}$ is a positive (resp. negative) basis for $T_x M$.
\end{tcbdfn}
\footnotetext{This is just a matter of convention, which we refer to as \emph{outward normal first}. Equivalently, one could define the induced orientation using the opposite convention, called \emph{outward normal last}.}
\begin{example}\label{ex:orientation}
 Following such a convention, the oriented cylinders above will induce opposite orientations on the two components of the boundary.

 \[
  \begin{tikzpicture}[every tqft/.style={transform shape}, rotate=90, scale=1.5]
   \pic[tqft,
    incoming boundary components=1,
    outgoing boundary components=1,
    draw,
    every outgoing lower boundary component/.style={draw},
    every incoming lower boundary component/.style={dashed,draw},
    every outgoing upper boundary component/.style={decorate,
      decoration={markings, mark=at position .5 with {\arrow{>}},},
     },
    every incoming upper boundary component/.style={decorate,
      decoration={markings, mark=at position .5 with {\arrowreversed{>}},},
     },
    name=a,
    anchor={(0,0)}
   ];
   \pic[tqft,
    incoming boundary components=1,
    outgoing boundary components=1,
    draw,
    every outgoing lower boundary component/.style={draw},
    every incoming lower boundary component/.style={dashed,draw},
    every outgoing upper boundary component/.style={decorate,
      decoration={markings, mark=at position .5 with {\arrowreversed{>}},},
     },
    every incoming upper boundary component/.style={decorate,
      decoration={markings, mark=at position .5 with {\arrow{>}},},
     },
    name=b,
    anchor={(0,-2.5)}
   ];
   \node at (2,-1) [color=ibmBlue] {$\circlearrowleft$};
   \draw [->, color=ibmBlue] (2.35,0) -- node [above=.15em,align=right, color=ibmBlue] {$n$} (2.35,.5);
   \draw [->, color=ibmBlue] (2.35,0) -- (2,.5);
   \draw [->, color=ibmBlue] (2.35,-2) -- node [below=.15em,align=right, color=ibmBlue] {$n$} (2.35,-2.5);
   \draw [->, color=ibmBlue] (2.35,-2) -- (2.7,-2.5);
   \node at (1.25,0) {$\mathbb{S}^1$};
   \node at (1.25,-2) {$-\mathbb{S}^1$};
   \node at (2,-6) [color=ibmBlue] {$\circlearrowright$};
   \draw [->, color=ibmBlue] (2.35,-5) -- node [above=.15em,align=right, color=ibmBlue] {$n$} (2.35,-4.5);
   \draw [->, color=ibmBlue] (2.35,-5) -- (2.7,-5.5);
   \draw [->, color=ibmBlue] (2.35,-7) -- node [above=.15em,align=right, color=ibmBlue] {$n$} (2.35,-7.5);
   \draw [->, color=ibmBlue] (2.35,-7) -- (2,-6.5);
   \node at (1.25,-7) {$\mathbb{S}^1$};
   \node at (1.25,-5) {$-\mathbb{S}^1$};
  \end{tikzpicture}
 \]
\end{example}

We are finally ready to give the following definition.

\begin{tcbdfn}[Oriented bordisms]
 Let $\Sigma_0$, $\Sigma_1$ be two oriented closed manifolds of dimension $n-1$. An \emph{$n$-dimensional oriented bordism}\index{oriented bordism} from $\Sigma_0$ to $\Sigma_1$ is a triple $(M, i_0, i_1)$ consisting of
 \begin{itemize}
  \item an $n$-dimensional oriented manifold $M$ with boundary,
  \item two orientation-preserving embeddings
        \[
         i_0 \colon \Sigma_0 \times [0,\varepsilon) \to M
          \qquad
          i_1 \colon \Sigma_1 \times (1-\varepsilon, 1] \to M
        \]
        definining an \emph{in-boundary} $(\partial M)_0 \defeq i_0(\Sigma_0, 0)$ and an \emph{out-boundary} $(\partial M)_1 \defeq i_1(\Sigma_1, 1)$, such that $\partial M = (\partial M)_0 \sqcup (\partial M)_1$.
 \end{itemize}
\end{tcbdfn}

\begin{example}\label{ex:bordism-orientation}
 Let us draw an example of a 2-dimensional bordism $B \colon \mathbb{S}^1 \to \mathbb{S}^1$. We consider a 2-dimensional oriented cylinder $M= \mathbb{S}^1 \times I$ and the corresponding embeddings $i_0$, $i_1$, defining its orientation as a bordism.

 Looking at the first picture in Example \ref{ex:orientation}, we notice how to define a bordism from $\mathbb{S}^1$ to itself, the cylinder we are considering has boundary equal to $\partial M = \mathbb{S}^1 \sqcup (-\mathbb{S}^1)$.
 \begin{figure}[H]
  \centering
  \begin{tikzpicture}[every tqft/.style={transform shape}]
   % M central
   \node at (1, -1) {$M = \mathbb{S}^1 \times I$};
   \pic[name=a, tqft cylinder, rotate=90, anchor=between outgoing 1 and 2, at={(0,0)},
    every outgoing boundary component/.style={draw, dashed},
    every incoming upper boundary component/.style={draw, dashed},
    cobordism edge/.style={draw}];
   \pic[tqft cylinder, rotate=90, anchor=outgoing boundary, at=(a-incoming boundary),
    every incoming upper boundary component/.style={draw, decoration={markings, mark=at position .5 with {\arrow{<}},}, postaction=decorate},
    cobordism edge/.style={draw},
    cobordism height=0.2cm, fill=ibmYellow, fill opacity=.6];
   \pic[tqft cylinder, rotate=90, anchor=incoming boundary, at=(a-outgoing boundary),
    every outgoing lower boundary component/.style={draw},
    every outgoing upper boundary component/.style={draw, decoration={markings, mark=at position .5 with {\arrow{>}},}, postaction=decorate},
    every outgoing boundary component/.style={fill=ibmBlue, fill opacity=.4},
    cobordism edge/.style={draw},
    cobordism height=0.2cm, fill=ibmBlue, fill opacity=.6];
   % left
   \draw [thick, right hook->, ibmYellow] (-3,0) -- node [above=.5em,align=center, color=ibmYellow] {$i_0$} (-1,0);
   \pic[tqft cylinder, rotate=90, at={(-4,0)}, cobordism height=0.2cm,
    every incoming upper boundary component/.style={draw, decoration={markings, mark=at position .5 with {\arrow{<}},}, postaction=decorate},
    fill=ibmYellow, fill opacity=.6,
    every incoming lower boundary component/.style={dashed, draw},
    every outgoing boundary component/.style={fill=ibmYellow, fill opacity=.4, dashed, draw},
    cobordism edge/.style={draw}];
   \node at (-4, -1) {$\mathbb{S}^1 \times [0, \varepsilon)$};
   % right
   \draw [thick, left hook->, ibmBlue] (5,0) -- node [above=.5em,align=center, color=ibmBlue] {$i_1$} (3,0);
   \pic[tqft cylinder, rotate=90, at={(6,0)}, cobordism height=0.2cm, fill=ibmBlue, fill opacity=.6,
    every incoming boundary component/.style={dashed, draw},
    every outgoing boundary component/.style={draw, fill=ibmBlue, fill opacity=.4},
    every outgoing upper boundary component/.style={draw, decoration={markings, mark=at position .5 with {\arrow{>}},}, postaction=decorate},
    cobordism edge/.style={draw}];
   \node at (6, -1) {$\mathbb{S} \times (1- \varepsilon, 1]$};
  \end{tikzpicture}
 \end{figure}

 The figure shows how the choice of intervals we made in the definition assures that the inclusion maps are both orientation-preserving.

 We can also see that the choice of in-boundaries and out-boundaries does not depend on the orientation of the manifold $M$, but on the ones of the copies of $\mathbb{S}^1$. For example, we could consider the same manifold $M$ and equip it with two in-boundaries and zero out-boundaries, as depicted below. In this case $i_0$ will be defined as the obvious inclusion on the component $\mathbb{S}^1 \times [0,\varepsilon)$ and as $i_0(x,t) = (x, 1-t)$ on the component $\overline{\mathbb{S}^1} \times [0, \varepsilon)$.

 \begin{figure}[H]
  \centering
  \begin{tikzpicture}[every tqft/.style={transform shape}]
   % M central
   \node at (1, -1) {$M = \mathbb{S}^1 \times I$};
   \pic[name=a, tqft cylinder, rotate=90, anchor=between outgoing 1 and 2, at={(0,0)},
    every outgoing boundary component/.style={draw, dashed},
    every incoming upper boundary component/.style={draw, dashed},
    cobordism edge/.style={draw}];
   \pic[tqft cylinder, rotate=90, anchor=outgoing boundary, at=(a-incoming boundary),
    every incoming upper boundary component/.style={draw, decoration={markings, mark=at position .5 with {\arrow{<}},}, postaction=decorate},
    cobordism edge/.style={draw},
    cobordism height=0.2cm, fill=ibmYellow, fill opacity=.6];
   \pic[tqft cylinder, rotate=90, anchor=incoming boundary, at=(a-outgoing boundary),
    every outgoing lower boundary component/.style={draw},
    every outgoing upper boundary component/.style={draw, decoration={markings, mark=at position .5 with {\arrow{>}},}, postaction=decorate},
    every outgoing boundary component/.style={fill=ibmYellow, fill opacity=.4},
    cobordism edge/.style={draw},
    cobordism height=0.2cm, fill=ibmYellow, fill opacity=.6];
   %left
   \draw [thick, right hook->, ibmYellow] (-3,0) -- node [above=.5em,align=center, color=ibmYellow] {$i_0$} (-1,0);
   \pic[tqft cylinder, rotate=90, at={(-4,0)}, cobordism height=0.2cm,
    every incoming upper boundary component/.style={draw, decoration={markings, mark=at position .5 with {\arrow{>}},}, postaction=decorate},
    fill=ibmYellow, fill opacity=.6,
    every incoming lower boundary component/.style={dashed, draw},
    every outgoing boundary component/.style={fill=ibmYellow, fill opacity=.4, dashed, draw},
    cobordism edge/.style={draw}];
   \pic[tqft cylinder, rotate=90, at={(-5,0)}, cobordism height=0.2cm,
    every incoming upper boundary component/.style={draw, decoration={markings, mark=at position .5 with {\arrow{<}},}, postaction=decorate},
    every incoming lower boundary component/.style={dashed, draw},
    fill=ibmYellow, fill opacity=.6,
    every outgoing boundary component/.style={fill=ibmYellow, fill opacity=.4, dashed, draw},
    cobordism edge/.style={draw}];
   \node at (-4.5, -1) {$(\mathbb{S}^1 \sqcup \overline{\mathbb{S}^1}) \times [0, \varepsilon)$};
   % right
   \draw [thick, left hook->, ibmBlue] (5,0) -- node [above=.5em,align=center, color=ibmBlue] {$i_1$} (3,0);
   \node at (6, 0) {$\varnothing$};
   \node at (6, -1) {$\varnothing \times (1-\varepsilon, 1]$};
  \end{tikzpicture}
 \end{figure}
\end{example}

\begin{remark}
 These examples let us notice why we asked for both the $n$-dimensional manifold and the $(n-1)$-dimensional manifolds to be oriented. Indeed, by specifying orientations on all manifolds, we are, in a way, fixing an in/out boundary distinction. For example, given the manifold $M = \mathbb{S}^1 \times I$ equipped with the orientation above, it would not be possible to define a bordism from $\mathbb{S}^1$ to $\overline{\mathbb{S}^1}$, since we would need to have $\partial M = \mathbb{S}^1 \sqcup (-\overline{\mathbb{S}^1}) = \mathbb{S}^1 \sqcup \mathbb{S}^1$.
\end{remark}
From now on, we will draw in-boundaries on the left and out-boundaries on the right, to avoid further confusion.
Thus, the previous cylinder representing a bordism $M \colon \mathbb{S}^1 \sqcup \overline{\mathbb{S}^1} \to \varnothing$ will be depicted as follows:

\[
 \begin{tikzpicture}[every tqft/.style={transform shape}]
  \pic[tqft, incoming boundary components=2, outgoing boundary components=0, draw, rotate=90, every lower boundary component/.style={dashed, draw}, boundary separation=1.25cm, cobordism height=2.5cm, anchor=incoming boundary 1, at={(0,0)}];
  \node at (-1,0) {$\overline{\mathbb{S}^1}$};
  \node at (-1,1.25) {$\mathbb{S}^1$};
 \end{tikzpicture}
\]

The role of the embeddings $i_0$, $i_1$ in the above definition is to put some sort of ``collar'' around the in-boundary and the out-boundary. Let us state this properly.

\begin{tcbdfn}[Collars \cite{smooth}]
 Let $M$ be a manifold with boundary. A neighbourhood of $\partial M$ is called a \emph{collar neighbourhood}\index{collar neighbourhood} if it is the image of a smooth embedding $\partial M \times [0, \varepsilon) \to M$ that restricts to the identification $\partial M \times \{0\} \to \partial M$.
\end{tcbdfn}
\begin{tcbthm}[Collar neighbourhood theorem]
 Let $M$ be a smooth manifold with nonempty boundary. Then $\partial M$ has a collar neighbourhood.
\end{tcbthm}

This guarantees that the notion of an oriented bordism is well defined.

\begin{tcbthm}[Gluing smooth manifolds along their boundaries]\label{th:gluing}
 Let $M$, $N$ be two manifolds with nonempty boundaries $\partial M$, $\partial N$. Suppose we have a diffeomorphism between the two boundaries and consider the topological pushout $M \sqcup_{\partial M \iso \partial N} N$. Such a topological manifold then has a smooth structure, compatible with the smooth structures on $M$ and $N$. If $M$, $N$ are both compact, $M \sqcup_{\partial M \iso \partial N} N$ is compact. If $M$, $N$ are both connected, $M \sqcup_{\partial M \iso \partial N} N$ is connected.
\end{tcbthm}

\begin{remark}
 While we do not give a formal proof of such a result, we observe that the smooth structure on the topological pushout is not unique and relies on the choice of collars.
 Our definition of bordisms with fixed collars, taken from \cite{moore2025tasi} and \cite{freed2019lectures}, is given precisely to specify this choice, making the glueing to be exactly the pushout in the proper category (which we can imagine as the category of ``manifolds with defined collars'').
 However, different choices of collars give rise to diffeomorphic smooth structures. 
\end{remark}

\begin{example}
 We illustrate such a procedure with the following example, considering bordisms of dimension 2. We take a bordism $(X, i_0, i_1)$ from $\Sigma_0 = \mathbb{S}^1 \sqcup \mathbb{S}^1$ to $\Sigma_1 = \mathbb{S}^1$ and a bordism $(X', i'_0, i'_1)$ from $\Sigma_1 = \mathbb{S}^1$ to $\Sigma_2 = \mathbb{S}^1$ and glue them along the common boundary $\Sigma_1$. Since our goal is to define a sort of ``composition'' we will always glue an out-boundary to an in-boundary.
 \begin{figure}[H]
  \centering
  \begin{tikzpicture}[every tqft/.style={transform shape}]
   \node at (-2.2, -.35) {$X$};
   \node at (2, 0.2) {$X'$};
   \node at (4.25,-.35) {become};
   \node at (8.5, -1) {$X \sqcup_{Y_1} X'$};
   \pic[name=a, tqft, offset=0.7, incoming boundary components=2,
    outgoing boundary components=1,
    rotate=90, anchor=outgoing boundary, at={(0,0)}, every upper boundary component/.style={draw, dashed},
    cobordism edge/.style={draw}];
   \pic[tqft cylinder, anchor=outgoing boundary, at=(a-incoming boundary 1), rotate=90,
    every incoming upper boundary component/.style={draw},
    cobordism edge/.style={draw},
    cobordism height=0.2cm, fill=ibmYellow, fill opacity=.6];
   \pic[tqft cylinder, anchor=outgoing boundary, at=(a-incoming boundary 2), rotate=90,
    every incoming upper boundary component/.style={draw},
    cobordism edge/.style={draw},
    cobordism height=0.2cm, fill=ibmYellow, fill opacity=.6];
   \pic[tqft cylinder, anchor=incoming boundary, at=(a-outgoing boundary), rotate=90,
    every outgoing lower boundary component/.style={draw},
    every outgoing upper boundary component/.style={draw},
    every outgoing boundary component/.style={fill=ibmBlue, fill opacity=.4},
    cobordism edge/.style={draw},
    cobordism height=0.2cm, fill=ibmBlue, fill opacity=.6];
   \pic[name=b, tqft cylinder, offset=.3, rotate=90, anchor=incoming boundary, at={(.6,-1.2)}, every upper boundary component/.style={draw, dashed},
    cobordism edge/.style={draw}];
   \pic[tqft cylinder, anchor=outgoing boundary, at=(b-incoming boundary), rotate=90,
    every incoming upper boundary component/.style={draw},
    cobordism edge/.style={draw},
    cobordism height=0.2cm, fill=ibmYellow, fill opacity=.6];
   \pic[tqft cylinder, anchor=incoming boundary, at=(b-outgoing boundary), rotate=90,
    every outgoing lower boundary component/.style={draw},
    every outgoing upper boundary component/.style={draw},
    every outgoing boundary component/.style={fill=ibmBlue, fill opacity=.4},
    cobordism edge/.style={draw},
    cobordism height=0.2cm, fill=ibmBlue, fill opacity=.6];
   \pic[name=c, tqft, offset=0.7, incoming boundary components=2,
    outgoing boundary components=1,
    rotate=90, anchor=outgoing boundary, at={(8,0)}, every upper boundary component/.style={draw, dashed},
    cobordism edge/.style={draw}];
   \pic[tqft cylinder, anchor=outgoing boundary, at=(c-incoming boundary 1), rotate=90,
    every incoming upper boundary component/.style={draw},
    cobordism edge/.style={draw},
    cobordism height=0.2cm, fill=ibmYellow, fill opacity=.6];
   \pic[tqft cylinder, anchor=outgoing boundary, at=(c-incoming boundary 2), rotate=90,
    every incoming upper boundary component/.style={draw},
    cobordism edge/.style={draw},
    cobordism height=0.2cm, fill=ibmYellow, fill opacity=.6];
   \pic[tqft cylinder, name=d, anchor=incoming boundary, at=(c-outgoing boundary), rotate=90,
    every outgoing upper boundary component/.style={draw},
    cobordism edge/.style={draw},
    cobordism height=0.2cm];
   \pic[tqft cylinder, name=e, anchor=incoming boundary, at=(d-outgoing boundary), rotate=90,
    cobordism edge/.style={draw},
    cobordism height=0.2cm];
   \pic[name=f, tqft cylinder, offset=.3, rotate=90, anchor=incoming boundary, at=(e-outgoing boundary), every upper boundary component/.style={draw, dashed},
    cobordism edge/.style={draw}];
   \pic[tqft cylinder, anchor=incoming boundary, at=(f-outgoing boundary), rotate=90,
    every outgoing lower boundary component/.style={draw},
    every outgoing upper boundary component/.style={draw},
    every outgoing boundary component/.style={fill=ibmBlue, fill opacity=.4},
    cobordism edge/.style={draw},
    cobordism height=0.2cm, fill=ibmBlue, fill opacity=.6];
  \end{tikzpicture}
 \end{figure}
\end{example}

More is actually true when considering the glueing of two oriented bordisms.

\begin{tcbthm}\label{th:collar_diffeo}
 Let $\Sigma$ be an out-boundary of a bordism $M_0$ and an in-boundary of a bordism $M_1$, and consider $M_0 \sqcup_\Sigma M_1$ the pushout through $\Sigma$ of the two topological manifolds. Let $\alpha$, $\beta$ be two smooth structures on $M_0 \sqcup_\Sigma M_1$, which both induce the original smooth structures on $M_0$ and $M_1$ (via pullback along the inclusion maps). Then there is a diffeomorphism $\phi \colon (M_0 \sqcup_\Sigma M_1, \alpha) \to (M_0 \sqcup_\Sigma M_1, \beta)$ such that its restriction on $\Sigma$ is the identity $\id_\Sigma$.
\end{tcbthm}

\section{A category of oriented boridsms}

Our goal is now to construct a category of oriented $n$-dimensional bordisms. The intuitive idea behind it is to take closed oriented $(n-1)$-dimensional manifolds as objects and oriented bordisms between them as morphisms. The proper definition, however, requires some more refinement.  We begin by addressing some technical issues. While equipping each bordism with an explicit choice of a collar allows us to properly define a glueing, this approach fails when trying to define a strict identity morphism. Given a closed oriented $(n-1)$-dimensional manifold $\Sigma$, a candidate for $\id_\Sigma$ is given by the cylinder $\Sigma \times [0,1]$ with some choice of collars. However, glueing it to a bordism $M$ with one of the boundaries equal to $\Sigma$ only gives us a manifold \emph{diffeomorphic} to $M$. By considering the underlying topological spaces, we easily understand that the only way to define a strict identity in this setting is to consider $\Sigma$ itself as a bordism, but this would not satisfy our definition, which requires it to be an $n$-dimensional manifold. Finally, we recall how, when glueing manifolds, different choices of collars give rise to different but diffeomorphic smooth structures. This motivates the following framework: we define the bordism category by taking as morphisms bordisms ``up to diffeomorphism''.

\begin{tcbdfn}[Equivalent bordisms]
 Let $(M, i_0, i_1)$, $(M', i'_0, i'_1)$ be two oriented bordisms, both from $\Sigma_0$ to $\Sigma_1$.
 We say $M$ and $M'$ are \emph{equivalent bordisms}\index{equivalent bordisms} if there exists an orientation-preserving diffeomorphism $\psi \colon M \to M'$ making the following diagram commute.
 \[
  \begin{tikzcd}
   & M \arrow[dd, dashed, "\psi"] & \\
   \Sigma_0 \times \{0\} \iso \Sigma_0 \arrow[ur, "i_0"] \arrow[dr, "i'_0", swap] && \Sigma_1 \iso \Sigma_1 \times \{1\} \arrow[ul, "i_1", swap] \arrow[dl, "i'_1"] \\
   & M'&
  \end{tikzcd}
 \]
\end{tcbdfn}

This clearly defines an equivalence relation between bordisms. We can then consider the equivalence classes $[(M, i_0, i_1)]$.

\begin{remark}
 In defining the equivalence class, we are in a way forgetting the collar choice we gave in the definition of a bordism.
 Indeed, when considering two bordisms $(M, i_0, i_1)$, $(M, j_0, j_1)$, we are asking for
 \[
  i_0(\Sigma_0, 0) = (\partial M)_0 = j_0(\Sigma_0, 0) \qquad j_0(\Sigma_1, 1) = (\partial M)_1 = j_1(\Sigma_1, 1)
 \]
 hence making the diagram commute by just choosing the identity map on $M$.
 From now on, when referring to an equivalence class of bordisms, we can forget the collar data and only consider the diffeomorphisms $\Sigma_0 \iso (\partial M)_0$ and $\Sigma_1 \iso (\partial M)_1$.
 We'll then just say that $M$ is a bordism from $\Sigma_0$ to $\Sigma_1$, without specifying further information.
\end{remark}

\begin{tcblemma}[Composition of cobordism classes]
 Given a bordism $M$ from $\Sigma_0$ to $\Sigma_1$ and a bordism $N$ from $\Sigma_1$ to $\Sigma_2$, we define their composition $MN$ from $\Sigma_0$ to $\Sigma_2$ as follows: we take any representative from each class, glue them and consider the equivalence class of the resulting glueing.
 This composition is well defined.
\end{tcblemma}

\begin{proof}
 Take the bordisms $(M, i_0, i_1)$, $(M', i'_0, i'_1)$ from the first equivalence class and $(N, i_0, i_1)$, $(N', i'_0, i'_1)$ from the second. This means we have two diffeomorphisms $\psi_M$, $\psi_N$ such that:
 \[
  \begin{tikzcd}
   & M \arrow[dd, "\psi_0"] & & N \arrow[dd, "\psi_1"] & \\
   \Sigma_0 \arrow[ur] \arrow[dr] & & \Sigma_{1} \arrow[ul] \arrow[ur] \arrow[dl] \arrow[dr] & & \Sigma_2 \arrow[ul] \arrow[dl] \\
   %\Sigma_0 \times \{0\} \iso \Sigma_0 \arrow[ur] \arrow[dr] & & \Sigma_1 \times \{1\} \iso \Sigma_1 \iso \Sigma_{1} \times \{0\} \arrow[ul] \arrow[ur] \arrow[dl] \arrow[dr] & & \Sigma_2 \iso \Sigma_2 \times \{1\} \arrow[ul] \arrow[dl] \\
   & M' & & N' &
  \end{tikzcd}
 \]
 We can then consider the gluings $(MN, i_0, j_1)$ and $(M'N', i'_0, j'_1)$.
 By taking the pushout of the two diffeomorphisms in the category of continuous maps, we get a homeomorphism $\psi \colon MN \to M'N'$.
 %\todo[inline]{come si comporta? preserva la struttura liscia? Se lo fa sistema il finale}
 Through such a homeomorphism, we can define a new smooth structure on $M'N'$, which by the previous theorem \ref{th:collar_diffeo} is diffeomorphic to the one induced by the glueing.
\end{proof}

\begin{tcbdfn}[A category of oriented bordisms]
 The category of oriented n-dimensional bordisms $\bord{n}$ \index{\bord{n}, the category of oriented $n$-bordisms} is defined as follows.
 \begin{itemize}
  \item The objects are closed oriented $(n-1)$-dimensional manifolds
  \item For any $\Sigma_0, \Sigma_1 \in \obj{\bord{n}}$, morphisms are the equivalence classes of bordisms $M \colon \Sigma_0 \to \Sigma_1$
  \item Composition of morphisms is obtained by glueing
  \item For each object $\Sigma$, the identity map $\id_\Sigma$ is given by the bordism $\Sigma \times [0,1]$
 \end{itemize}
\end{tcbdfn}

% def disjoint union of bordisms
\begin{dfnx*}[Disjoint union of bordisms]
 Given two bordisms $M \colon \Sigma_0 \to \Sigma_1$ and $N \colon \Sigma'_0 \to \Sigma'_1$, their disjoint union $M \disj N$ naturally defines a bordism from $\Sigma_0 \disj \Sigma'_0$ to $\Sigma_1 \disj \Sigma'_1$.
 This operation is precisely the coproduct in the category of smooth manifolds, equipped with the unique orientation agreeing with the ones on $M$ and $N$.
\end{dfnx*}

\begin{tcbprp}
 The category $\bord{n}$ has a monoidal structure given by the disjoint union of manifolds (and the initial object, being the empty manifold $\varnothing$).
\end{tcbprp}

\begin{proof}
 Since the disjoint union is the coproduct in the category $\bord{n}$, the result follows from lemma \ref{prp:cocartesian-monoidal}.
\end{proof}

\begin{tcbprp}[Embedding diffeomorphisms in the bordism category]
 Any diffeomorphism of $(n-1)$-dimensional manifolds $\Sigma_0$, $\Sigma_1$ define an equivalence class of (invertible) bordisms $M \colon \Sigma_0 \to \Sigma_1$.
\end{tcbprp}

\begin{proof}
 Let $\Sigma$ be a closed (oriented) $(n-1)$-dimensional manifold and $f \colon \Sigma \to \Sigma$ a diffeomorphism. By considering $X = \Sigma \times [0,1]$ equipped with the usual orientation and the obvious inclusions $i_0 \colon \Sigma \iso \Sigma \times \{0\}\hookrightarrow X$ and $i_1 \colon \Sigma \iso \Sigma \times \{1\} \hookrightarrow X$ we define a bordism (or better, an equivalence class of bordisms) $X \colon \Sigma \to \Sigma$. The same procedure can be applied when considering two diffeomorphic $(n-1)$-dimensional manifolds and the diffeomorphism $f \colon \Sigma_0 \to \Sigma_1$ between them. We can indeed consider the $n$-dimensional manifold $X = \Sigma_0 \times I$ and the inclusions $i_0 \colon \Sigma_0 \to X$, $i_1 \colon \Sigma_1 \iso \Sigma_0 \to X$ as in the previous case.
\end{proof}

We also notice that given two composable diffeomorphisms $f \colon \Sigma_0 \to \Sigma_1$, $g \colon \Sigma_1 \to \Sigma_2$, we have the corresponding bordisms $X_f$, $X_g$ and their composition $X_fX_g$.

\begin{tcbprp}
 Two diffeomorphisms $f \colon \Sigma_0 \to \Sigma_1$, $g \colon \Sigma_0 \to \Sigma_1$ give rise to the same bordism class if and only if they are smoothly homotopic.
\end{tcbprp}
\begin{proof}
 Recall that $f \colon \Sigma_0 \to \Sigma_1$, $g \colon \Sigma_0 \to \Sigma_1$ are (smoothly) homotopic if there exists a smooth map $\phi \colon \Sigma_0 \times [0,1] \to \Sigma_1$ such that $\phi(x, 0) = f(x)$ and $\phi(x, 1) = g(x)$. Equivalently, we are asking for the following diagram to be commutative.
 \[
  \begin{tikzcd}
   \Sigma_0 \arrow[r] \arrow[dr, "f", swap] & \Sigma_0 \times I \arrow[d, "\phi"] & \Sigma_0 \arrow[l] \arrow[dl, "g"] \\
   & \Sigma_1 &
  \end{tikzcd}
 \]
 By composition with the inclusion $\Sigma_1 \hookrightarrow \Sigma_1 \times I$ and with $g^{-1} \colon \Sigma_1 \to \Sigma_0$ we have
 \[
  \begin{tikzcd}
   \Sigma_0 \arrow[r] \arrow[dr, "f", swap] & \Sigma_0 \times I \arrow[d, "\phi"] & \Sigma_0 \arrow[l] \arrow[dl, "g"] & \Sigma_1 \arrow[l, "g^{-1}", ibmBlue] \\
   & \Sigma_1 \arrow[d, ibmBlue] & & \\
   & \Sigma_1 \times I & &
  \end{tikzcd}
 \]
 which gives the following commutative diagram.
 \[
  \begin{tikzcd}
   & \Sigma_0 \times I \arrow[dd, "\phi"] & \\
   \Sigma_0 \arrow[ru, "\id"] \arrow[rd, "f", swap] & & \Sigma_1 \arrow[lu, "g^{-1}", swap] \arrow[ld, "\id"] \\
   & \Sigma_1 \times I &
  \end{tikzcd}
 \]
 Even if the naming of the arrows is a bit sloppy, this defines the equivalence between the two bordisms.
 To prove the converse we take two equivalent bordisms $\Sigma_0 \times I$ and $\Sigma_1 \times I$ and the orientation-preserving diffeomorphism $\psi \colon \Sigma_0 \times I \to \Sigma_1 \times I$.
 \[
  \begin{tikzcd}
   & \Sigma_0 \times I \arrow[dd, dashed, "\psi"] & \\
   \Sigma_0 \arrow[ur, "i_0"] \arrow[dr, "i'_0", swap] && \Sigma_1 \arrow[ul, "i_1", swap] \arrow[dl, "i'_1"] \\
   & \Sigma_1 \times I &
  \end{tikzcd}
 \]
 By composition with the projection $\Sigma_1 \times I \to \Sigma_1$, we get the diagram defining the homotopy between $f$ and $g$.
\end{proof}

These considerations allow us to define the following class of bordisms, ensuring it is not equal to the one arising from $\id \sqcup \id$.

\begin{dfnx*}[The twist bordism]
 The twist diffeomorphism of manifolds $\sigma \colon \Sigma \disj \Sigma' \to \Sigma' \disj \Sigma$ defines a cobordism in $\bord{n}$ which we'll denote as
 $T_{\Sigma,\Sigma'} \colon \Sigma \disj \Sigma' \to \Sigma' \disj \Sigma$.
\end{dfnx*}

% maybe add T natural in Bord

\begin{tcbprp}
 The category $\bord{n}$ has a symmetric monoidal structure $(\bord{n}, \disj, \varnothing, T)$
\end{tcbprp}

\section{Topological Quantum Field Theories}

\begin{tcbdfn} \label{dfn:tqft}
 An \emph{$n$-dimensional topological quantum field theory} \index{topological quantum field theory} is a symmetric monoidal functor from $(\bord{n}, \disj, \varnothing, T)$ to $(\mathbf{Vect}_\Bbbk, \tensor, \Bbbk, \sigma)$
\end{tcbdfn}

Historically, the first explicit axiomatization of TQFTs was given by Atiyah in 1988 \cite{Atiyah1988}.
Note that in Atiyah we do not yet encounter the explicit notion of bordisms as we defined them.

Although the interest in the topic mainly comes from physics, we will, as mentioned in the introduction, ignore that perspective entirely.

We here give a slightly refined version (which can be found in \cite{kock2003frobenius}) of the original axioms, stated in terms of bordisms.

\begin{tcbdfn}[Axiomatization of a TQFT]
 An \emph{n-dimensional topological quantum field theory over a field $\Bbbk$} \index{topological quantum field theory} consists of:
 \begin{itemize}
  \item a $\Bbbk$-vector space $Z(\Sigma)$ associated to each closed oriented $(n-1)$-dimensional manifold $\Sigma$
  \item a $\Bbbk$-linear map $Z(M) \colon Z(\Sigma_0) \to Z(\Sigma_1)$ associated to each $n$-dimensional bordism $M$ from $\Sigma_0$ to $\Sigma_1$
 \end{itemize}
 satisfying the following axioms:
 \begin{enumerate}[label=A\arabic*]
  \item \label{tqft:a1} Two equivalent bordisms $M \iso N$ have the same image through $Z$, namely $Z(M)=Z(N)$
  \item \label{tqft:a2} The cylinder bordism $M = \Sigma \times I \colon \Sigma \to \Sigma$ is mapped to the identity
        \[ Z(\Sigma \times I) = \id_{Z(\Sigma)} \colon Z(\Sigma) \to Z(\Sigma) \]
  \item \label{tqft:a3} The glueing of two bordisms is mapped to the composition of their images, meaning that for two composable morphisms $M$, $N$, we have
        \[ Z(MN) = Z(N) \circ Z(M) \]
  \item \label{tqft:a4} The disjoint union of two bordisms is mapped to the tensor product of their images, meaning
        \[Z(M \disj N) = Z(M) \tensor Z(N) \]
  \item \label{tqft:a5} The empty manifold $\varnothing$ is mapped to the ground field $\Bbbk$. It follows from A\ref{tqft:a2} that the empty bordism $\varnothing \times I$ is sent to the identity $\id_\Bbbk$.
 \end{enumerate}
\end{tcbdfn}

\begin{remark}
 It seems natural to interpret these axioms in the language of category theory.
 Indeed, Axiom \ref{tqft:a1} well defines a map from $\bord{n}$ to $\mathbf{Vect}_\Bbbk$ and Axioms \ref{tqft:a2} and \ref{tqft:a3} guarantee such a map is really a functor.
 Axioms \ref{tqft:a4} and \ref{tqft:a5} preserve the monoidal structure. To get a \emph{symmetric} monoidal functor, as defined in \ref{dfn:tqft}, we would have to add a sixth axiom:
 \begin{enumerate}
  \item[(A6)] The twist bordism is mapped to the twist map of vector spaces, meaning
        \[ Z(T_{M,N}) = \sigma_{Z(M),Z(N)} \]
 \end{enumerate}
\end{remark}

\begin{dfnx*}[A brief digression on dualizability\index{dualizability}]
 We now leave some space to discuss an important property regarding TQFTs, starting with what is known as the \emph{snake decomposition of a cylinder}\index{snake decomposition}.

 Take any closed $(n-1)$-dimensional manifold $\Sigma$ and consider the correspondig bordism $\Sigma \times I$. There are various ways in which we can decompose it. We can just cut it in the center
 \[
  \begin{tikzpicture}[every tqft/.style={transform shape}, scale=0.75, tqft/boundary separation=1.5cm]
   \pic[tqft cylinder, name=a, draw, rotate=90, anchor=incoming boundary, draw, every lower boundary component/.style={dashed, draw}];
   \pic[tqft cylinder, name=b, draw, rotate=90, anchor=incoming boundary, at=(a-outgoing boundary), draw, every outgoing lower boundary component/.style={draw}];
   \node at (-.5, 0) {$\Sigma$};
   \node at (4.5, 0) {$\Sigma$};
   %\draw [line width=.5mm, -{Stealth[length=2mm]}, opacity=0.75](0.0,0) -- (.5,0);
   % \draw [line width=.5mm, -{Stealth[length=2mm]}, opacity=0.75](2,0) -- (2.5,0);
   %\draw [line width=.5mm, -{Stealth[length=2mm]}, opacity=0.75](4,0) -- (4.5,0);
  \end{tikzpicture}
 \]
 or we can bend it before cutting it, obtaining something like
 \[
  \begin{tikzpicture}[every tqft/.style={transform shape}, scale=0.75]
   \pic[tqft cylinder, name=a, draw, rotate=90, anchor=incoming boundary, every lower boundary component/.style={dashed, draw}];
   \pic[tqft, name=b, incoming boundary components=2, outgoing boundary components=0, draw, rotate=90,
    anchor=incoming boundary 2, at=(a-outgoing boundary), every lower boundary component/.style={dashed, draw}, boundary separation=1.25cm, cobordism height=2.5cm];
   \pic[tqft, name=c, incoming boundary components=0, outgoing boundary components=2, draw, rotate=90,
    anchor=outgoing boundary 2, at=(b-incoming boundary 1), every lower boundary component/.style={dashed, draw}, boundary separation=1.25cm, cobordism height=2.5cm];
   \pic[tqft cylinder, draw, rotate=90, anchor=incoming boundary, at=(c-outgoing boundary 1), every outgoing lower boundary component/.style={draw}];
   \node at (-.5, 0) {$\Sigma$};
   \node at (4.5, -2.5) {$\Sigma$};
  \end{tikzpicture}
 \]
 Notice how, after the cutting, the middle copy of $\Sigma$ is equipped with its opposite orientation for reasons we discussed in Example \ref{ex:bordism-orientation}. We will denote it as $\overline{\Sigma}$. This type of decomposition is called the \emph{snake decomposition} and can be written as a composition of arrows as
 \[
  \Sigma \iso \Sigma \sqcup \varnothing \xrightarrow{\id_\Sigma \sqcup H} \Sigma \sqcup \overline{\Sigma} \sqcup \Sigma \xrightarrow{N \sqcup \id_\Sigma} \varnothing \sqcup \Sigma \iso \Sigma
 \]
 where $N \colon \Sigma \sqcup \overline{\Sigma} \to \varnothing$ and $H \colon \varnothing \to \overline{\Sigma} \sqcup \Sigma$ are the ``bent cylinders''. We can then state that:
 \[
  \id_\Sigma \iso (N \sqcup \id_\Sigma) \circ (id_\Sigma \sqcup H)
 \]
 The same is true when taking the dual snake (i.e. the one obtained from a bordism $\overline{M} \colon \overline{\Sigma} \to \overline{\Sigma}$). In that case we'd have that $\id_{\overline{\Sigma}}$ is isomorphic to:
 \[
  \begin{tikzpicture}[every tqft/.style={transform shape}, scale=0.75, baseline=0.9375cm]
   \pic[tqft cylinder, name=a, draw, rotate=90, anchor=incoming boundary, every lower boundary component/.style={dashed, draw}];
   \pic[tqft, name=b, incoming boundary components=2, outgoing boundary components=0, draw, rotate=90,
    anchor=incoming boundary 1, at=(a-outgoing boundary), every lower boundary component/.style={dashed, draw}, boundary separation=1.25cm, cobordism height=2.5cm];
   \pic[tqft, name=c, incoming boundary components=0, outgoing boundary components=2, draw, rotate=90,
    anchor=outgoing boundary 1, at=(b-incoming boundary 2), every lower boundary component/.style={dashed, draw}, boundary separation=1.25cm, cobordism height=2.5cm];
   \pic[tqft cylinder, draw, rotate=90, anchor=incoming boundary, at=(c-outgoing boundary 2), every outgoing lower boundary component/.style={draw}];
   % \draw [line width=.5mm, -{Stealth[length=2mm]}, opacity=0.75](0,0) -- (.5,0);
   % \draw [line width=.5mm, -{Stealth[length=2mm]}, opacity=0.75](2,0) -- (2.5,0);
   % \draw [line width=.5mm, -{Stealth[length=2mm]}, opacity=0.75](2,-1.25) -- (1.5,-1.25);
   % \draw [line width=.5mm, -{Stealth[length=2mm]}, opacity=0.75](2,-2.5) -- (2.5,-2.5);
   % \draw [line width=.5mm, -{Stealth[length=2mm]}, opacity=0.75](4,-2.5) -- (4.5, -2.5);
   \node at (-.5, 0) {$\overline{\Sigma}$};
   \node at (4.5, 2.5) {$\overline{\Sigma}$};
  \end{tikzpicture}
  \hspace{2em}
  (\id_{\overline{\Sigma}} \sqcup N) \circ (H \sqcup \id_{\overline{\Sigma}}) \colon \overline{\Sigma} \to \overline{\Sigma} \sqcup \Sigma \sqcup \overline{\Sigma} \to \overline{\Sigma}
 \]
 An interesting result appears when evaluating a TQFT on these bordisms. Considering any $Z \colon \bord{n} \to \mathbf{Vect}_\Bbbk$ and naming $V = Z(\Sigma)$, $\overline{V} = Z(\overline{\Sigma})$ we have
 \[
  Z(\id_\Sigma) = \id_V \colon V \to V
  \hspace{3em}
  Z(N) = \beta \colon V \tensor \overline{V} \to \Bbbk
  \hspace{3em}
  Z(H) = \gamma \colon \Bbbk \to \overline{V} \tensor V
 \]
 Through functoriality of $Z$, the isomorphisms above define the following linear maps:
 \[
  \id_V = (\beta \tensor \id_V) \circ (\id_V \tensor \gamma) \colon V \iso V \tensor \Bbbk \to V \tensor \overline{V} \tensor V \to \Bbbk \tensor V \iso V
 \]
 \[
  \id_{\overline{V}} = (\id_{\overline{V}} \tensor \beta) \circ (\gamma \tensor \id_{\overline{V}}) \colon \overline{V} \iso \Bbbk \tensor \overline{V} \to \overline{V} \tensor V \tensor \overline{V} \to \overline{V} \tensor \Bbbk \iso \overline{V}
 \]
 with which we can state and prove the following proposition.
 \begin{tcbprp}\label{prp:finite-dimensional}
  The image vector spaces in a TQFT are necessarily of finite dimension.
 \end{tcbprp}
 \begin{proof}
  Let $\gamma(1_\Bbbk) = \sum_{i=1}^n \overline{v}_i \tensor v_i$ for $\overline{v}_i \in \overline{V}$ and $v_i \in V$.
  Evaluating the composition $(\beta \tensor \id_V) \circ (\id_V \tensor \gamma)$ on a vector $v \in V$ gives:
  \[
   v \mapsto v \tensor \sum_{i=1}^n \overline{v}_i \tensor v_i \mapsto \sum_{i=1}^n \beta(v \tensor \overline{v}_i) v_i
  \]
  Since this composition is equal to $\id_V$ we have, for every $v \in V$, the equivalence $v = \sum_{i=1}^n \beta(v \tensor \overline{v}_i) v_i$. This is equal to saying that every $v \in V$ can be written as a $\Bbbk$-linear combination of $\{v_1, \dots, v_n\}$, hence $V$ is a vector space of finite dimension.
  By evaluating the other composition in $\overline{v}$ and applying the same argument, we conclude that $\overline{V}$ is also finite-dimensional.
 \end{proof}

 These maps further imply a canonical duality between the two spaces. One can prove that $\overline{V}$ can be identified with the dual of $V$.
 \begin{tcbprp}
  Let $\Sigma$ be a closed $(n-1)$-manifold and $Z \colon \bord{n} \to \mathbf{Vect}_\Bbbk$ a TQFT. Let $V = Z(\Sigma)$ and $\overline{V} = Z(\overline{\Sigma})$. Then:
  \begin{enumerate}
   \item $\overline{V}$ is canonically isomorphic to $V^\ast$
   \item $V$ is canonically isomorphic to $\overline{V}^\ast$
  \end{enumerate}
 \end{tcbprp}
\end{dfnx*}
We omit the proof of such a result. Instead, we notice how in Atiyah's original axiomatization, this was an explicit axiom:
\begin{enumerate}
 \item[(A7)] The image vector space $V$ of a closed manifold $\Sigma$ comes equipped with a nondegenerate pairing with $\overline{V} = Z (\overline{\Sigma})$
\end{enumerate}
These observations are fundamental for understanding the core structure of the category we are working within. Indeed, they are a concrete instance of the categorical concept of \emph{dualizability}, the main definitions of which where given in Section \ref{sec:rigid}.

We now give a couple of examples of some simple TQFTs.

\begin{tcbex}[A trivial TQFT]
 We can define a \emph{trivial TQFT} by assigning a field $\Bbbk$ to every $(n-1)$-dimensional closed manifold and $\id_\Bbbk$ to every $n$-dimensional bordism.
\end{tcbex}

\begin{tcbex}[A 2-dimensional TQFT computing the genus]
 Consider a $\Bbbk$-vector space $V = \langle v_1, v_2 \rangle$ generated by two vectors. We can define a 2-dimensional TQFT $Z \colon \bord{2} \to \mathbf{Vect}_\Bbbk$ by the following assigments:
 \par
 The circle $\mathbb{S}^1$ is sent to the vector space $V$.
 \par
 \begin{align*}
  Z(\bordcap) \colon \Bbbk & \to V       & Z(\bordcocap) \colon V & \to \Bbbk \\
  1_\Bbbk                  & \mapsto v_1 & v_0                    & \mapsto 0 \\
                           &             & v_1                    & \mapsto 1
 \end{align*}
 \begin{align*}
  Z(\bordpants) \colon V \tensor V & \to V       & Z(\bordcopants) \colon V & \to V \tensor V                           \\
  v_0 \tensor v_0                  & \mapsto v_1 & v_0                      & \mapsto v_0 \tensor v_1 + v_1 \tensor v_0 \\
  v_0 \tensor v_1                  & \mapsto v_0 & v_1                      & \mapsto v_1 \tensor v_1 + v_0 \tensor v_0 \\
  v_1 \tensor v_0                  & \mapsto v_0 &                          &                                           \\
  v_1 \tensor v_1                  & \mapsto v_1 &                          &
 \end{align*}
 As we will prove in the end, defining $Z$ on these four bordisms is more than enough to get a complete definition of the TQFT. Now, let us understand through some explicit computations what this functor has to do with the genus of surfaces.
 Take for example the sphere $\mathbb{S}^2$, which can be seen as $\begin{tikzpicture}[every tqft/.style={transform shape}, scale=0.5, baseline=-.25ex]
   \pic[tqft cap, name=a, rotate=90, draw, every outgoing lower boundary component/.style={dashed, draw}];
   \pic[tqft cup, rotate=90, draw, anchor=incoming boundary, at=(a-outgoing boundary), every lower boundary component/.style={dashed, draw}];
  \end{tikzpicture}
 $. Then we have
 \[
  Z(\mathbb{S}^2) \colon 1_\Bbbk \mapsto v_1 \mapsto 1 \, (=2^0)
 \]
 Similarly, a torus $\mathbb{T}$ can be seen as $\begin{tikzpicture}[every tqft/.style={transform shape}, scale=0.35, baseline=-.25ex, tqft/boundary separation=1.5cm]
   \pic[tqft cap, name=a, rotate=90, draw, every outgoing lower boundary component/.style={dashed, draw}];
   \pic[tqft pair of pants, name=b, rotate=90, draw, anchor= incoming boundary, at=(a-outgoing boundary), every outgoing lower boundary component/.style={dashed, draw}];
   \pic[tqft reverse pair of pants, name=c, rotate=90, draw, anchor= incoming boundary 1, at=(b-outgoing boundary 1), every outgoing lower boundary component/.style={dashed, draw}];
   \pic[tqft cup, rotate=90, draw, anchor=incoming boundary, at=(c-outgoing boundary), every lower boundary component/.style={dashed, draw}];
  \end{tikzpicture}$, which means we'll compute
 \[
  Z(\mathbb{T}) \colon 1_\Bbbk \mapsto v_1 \mapsto v_0 \tensor v_0 + v_1 \tensor v_1 \mapsto v_1 + v_1 \mapsto 2 \, (= 2^1)
 \]
 Consider now a surface with two holes $2\mathbb{T}^2$ which we can draw by glueing the above bordisms as $\begin{tikzpicture}[every tqft/.style={transform shape}, scale=0.35, baseline=-.25ex, tqft/boundary separation=1.5cm]
   \pic[tqft cap, name=a, rotate=90, draw, every outgoing lower boundary component/.style={dashed, draw}];
   \pic[tqft pair of pants, name=b, rotate=90, draw, anchor= incoming boundary, at=(a-outgoing boundary), every outgoing lower boundary component/.style={dashed, draw}];
   \pic[tqft reverse pair of pants, name=c, rotate=90, draw, anchor= incoming boundary 1, at=(b-outgoing boundary 1), every outgoing lower boundary component/.style={dashed, draw}];
   \pic[tqft pair of pants, name=d, rotate=90, draw, anchor= incoming boundary, at=(c-outgoing boundary), every outgoing lower boundary component/.style={dashed, draw}];
   \pic[tqft reverse pair of pants, name=e, rotate=90, draw, anchor= incoming boundary 1, at=(d-outgoing boundary 1), every outgoing lower boundary component/.style={dashed, draw}];
   \pic[tqft cup, rotate=90, draw, anchor=incoming boundary, at=(e-outgoing boundary), every lower boundary component/.style={dashed, draw}];
  \end{tikzpicture}$. We get
 \[
  Z(2\mathbb{T}^2) \colon 1_\Bbbk \mapsto v_1 \mapsto v_0 \tensor v_0 + v_1 \tensor v_1 \mapsto 2v_1 \mapsto 2(v_0 \tensor v_0 + v_1 \tensor v_1) \mapsto 4v_1 \mapsto 4 \, (= 2^2)
 \]
 We see how we have defined a mapping $Z$ that assigns to each closed two-dimensional surface $\Sigma$ a linear map $Z(\Sigma) \colon \Bbbk \to \Bbbk$ that, evaluated at $1_\Bbbk$, returns $2^g$, where $g$ is the genus of the surface $\Sigma$.
\end{tcbex}

\subsection{A category of $n$TQFTs}

Remembering how we ended our first chapter and having just seen how $n$TQFT correspond to symmetric monoidal functors from $(\bord{n}, \disj, \varnothing, T)$ to $(\mathbf{Vect}_\Bbbk, \tensor, \Bbbk, \sigma)$, we can define a category of $n$-dimensional topological quantum field theories as follows.

\begin{tcbdfn}
 The category of $n$-dimensional TQFTs\index{$n\mathbf{TQFT}_\Bbbk$, the category of $n$-dimensional TQFTs} is the functor category
 \[
  \mathbf{nTQFT}_\Bbbk \coloneq \mathbf{SymMonCat}(\bord{n},\mathbf{Vect}_\Bbbk)
 \]
\end{tcbdfn}

Arrows in such a category will then be monoidal natural transformations between such functors. More precisely, given two $n$-dimensional TQFTs (i.e two symmetric monoidal functors $Z \colon \bord{n} \to \mathbf{Vect}_\Bbbk$, $Z' \colon \bord{n} \to \mathbf{Vect}_\Bbbk$), a natural transformation between them is then defined by $\Bbbk$-linear maps $Z(\Sigma) \to Z'(\Sigma)$ for $\Sigma$ object in $\bord{n}$.

%%% Local Variables:
%%% mode: LaTeX
%%% TeX-master: "../main"
%%% End:


% Capitolo 3
\chapter{The two dimensional case}
\label{chap:bord2}

\section{Generators and relations}

We now consider the category $\bord{2}$.
Recall that its objects are closed oriented 1-dimensional manifold and its arrows are the equivalence classes of oriented 2-dimensional bordisms.
Our goal is to describe its elementary pieces and how they interact with each other. 
To do so, we are going to define a \emph{normal form} of such bordisms, which will facilitate their comparison.
%This construction relies on some results from Morse theory, which we now briefly recall. 

We begin by adressing the existence of a bordism between closed oriented 1-dimensional manifolds.
It is a known result that all closed connected one dimensional manifolds are diffeomorphic to a circle $\mathbb{S}^1$. 
Any object in $\bord{2}$ is then diffeomorphic to a disjoint union of circles.
This doesn't depend on the orientation (either clockwise or counter-clockwise) we define on the circles.
Indeed, given two copies of $\mathbb{S}^1$ with opposite orientations, we can always define, by reflection, an orientation preserving diffeomorphism between them.



\backmatter


%Bibliografia
\printbibliography[
heading=bibintoc,
title={Bibliography}
]

\printindex
\end{document}
